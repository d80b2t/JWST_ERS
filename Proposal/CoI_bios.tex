\subsection*{Dr. Nicholas Ross}
P.I. Dr. Nic Ross is a deep believer in delivering science-enabling
products, including datasets, catalogs, analysis codes, plots,
algorithms and where possible computational resources to the wide
astronomical community.  As such, the call for delivering
science-enabling products by the release of the Cycle 2 Call for
Proposals (September 2019) is fully inline with his scientific
practice.

\smallskip \smallskip
\noindent
Ross has being developing and buidling up his GitHub Repositories over
the last year or so, \href{https://github.com/d80b2t}{\tt
github.com/d80b2t} and indeed now does all his analysis and paper
writing on GitHub.
%%  \smallskip \smallskip \noindent
Ross will devote a considerable amount of his personal research time
(and due to his STFC ERF has 100\% FTE for research) to leading the
developement and timely prodcution of the ERS ERQ science-enabling
products.


\subsection*{Dr. David Rosario} 
Co-PI Dr. David Rosario is awesome and also loves to write code. ;-)


\subsection*{Prof. David Alexander} 
Prof. Alexander is an expert in high-$z$ obsured AGN.  He will use his
considerable {\it Spitzer IRS} expereince to help test our MIRI MRS
data-analysis toolkit.


\subsection*{Dr. Rachael Alexandroff} 
Dr. Alexandroff is an leading expert on the ERQ population.  She will
bring to bear her now considerable and recent data analysis (long-slit
optical, polarimerty, radio) data analysis experience to build our
MIRI MRS data-analysis toolkit.


\subsection*{Dr. Manda Banerji}
Dr. Banerji is a Royal Society University Research Fellow and has
extensive experience in studying populations of obscured, infrared
luminous quasars as well as high-redshift quasars. She has
successfully applied for multi-wavelength follow-up time for these
populations on facilities such as XMM-Newton, VLT, JCMT, ALMA and
VLA.


\iffalse
\subsection*{Prof. Beth Biller}
Prof. Biller is an expert in infrared coronagraphic observations. 
While we do not intend to use the MIRI coronagraphs in this proposal, 
longer term observations would potentially involve observing the ERQs
with the Lyot or 4QPM if this became appropriate and technically feasible. 
\fi

\subsection*{Prof. Niel Brandt}


\subsection*{Prof. Xiaohui Fan}
Prof. Fan is a leader in surveys of high-redshift quasars and
reionization. He has extensive experience in studying quasars and
their host galaxies with {\it HST} and {\it Spitzer}.


\subsection*{Prof. Fred Hamann}


\subsection*{Prof. Dale Kocevski}
{\it We are also asking for the appriopriate level of post-doctoroal support for Prof. Kocevski.}


\iffalse
\subsection*{Prof. Linhua Jiang}
\fi


\subsection*{Dr. Stephanie LaMassa}
Dr. LaMassa is currently at the STScI and is already involved with the
documentation efforts there. As such, Dr. LaMassa will help with those
efforts, along with writing code and potentially leading follow-up
where approriate. She will also be a natural link to the direct
efforts of the Space Telsecop Science Institute.


\subsection*{Prof. Andy Lawrence}
Prof. Lawrence is the Regius Professor of Astronomy at the University
of Edinburgh. He has long history of research on Active Galaxies at
X-ray, optical, and IR wavelengths, as well as observational cosmology
and ultraluminous IR galaxies. He is the PI of the UKIDSS survey, has
extensive experience in managing astronomical software delivery in
wide field astronomy, and was one of the originators of the
International Virtual Observatory Alliance.


\subsection*{Dr. Chelsea MacLeod}
Dr. MacLeod has extensive experience analysing the time variability of
quasars and a strong foundation for working with survey data.  Using
SDSS data, Dr. MacLeod characterized the optical variability of
quasars in a sample many times larger than ever previously attempted.
By including Pan-STARRS data for SDSS quasars, she compiled a sample
of extremely variable quasars and is leading a spectroscopic followup
campaign in order to analyse their spectroscopic variability.

\smallskip \smallskip \noindent
Starting in April 2016 she has worked within the SDSS-IV collaboration
as part of the Time Domain Spectroscopic Survey (TDSS), a subprogram
of eBOSS that is obtaining optical spectra of time variable sources.
Dr. MacLeod is currently leading the TDSS quasar target selection and
is a co-chair of the Quasar Science Working Group of SDSS-IV
eBOSS. Dr. MacLeod will draw from her experience in the optical to
assist in the observational followup of quasars in the IR.


\subsection*{Dr. James Mullaney}
Dr. Mullaney has extensive experience in the analysis and interpretation of infrared observations of AGNs both in the local and high redshift Universe. The infrared SED templates and fitting code he developed have become a standard for analysing the infrared emission of AGNs and will be used extensively during this project. 

\subsection*{Prof. Adam Myers}
Prof. Myers is an expert on the statistical analysis of reddened,
obscured and optically luminous quasars. He has co-authored many
well-cited publications on targeting quasars, quasar clustering,
high-redshift and unusual quasars, and quasars in the time
domain. Prof. Myers has made follow-up observations of quasars, and
other objects, at telescopes on five continents. His work has been
funded multiple times by the NSF and NASA, including via space
telescope programs such as those for {\it Chandra} and {\it Spitzer}. He has
served on time allocation committees for GALEX and the {\it HST}. 

\smallskip \smallskip
\noindent
Prof. Myers has also worked extensively in large survey
collaborations, often in formal management roles. He is an Architect
of SDSS-III and SDSS-IV, was the quasar target selection lead for the
SDSS-IV/eBOSS survey, is the Level 3 Target Selection Manager for the
Dark Energy Spectroscopic Instrument (DESI) and is the documentation
and website lead for the Legacy Surveys
(http://legacysurvey.org). {\it Prof. Myers is a strong advocate for
transparent and reproducible science. For example, as part of his work
on DESI, he has contributed over 10,000 lines of code to publicly
visible github repositories.}


\subsection*{Dr. Jessie Runnoe}
Dr. Runnoe is an expert on quasar central engines at radio through
X-ray wavelengths.  Drawing on her vast observational experience, she
will contribute to the development of the MIRI MRS data-analysis
toolkit and assist with follow-up observations of the ERQ core sample.
She will be part of the Core Coding and Observational Follow-up
groups.


\subsection*{Prof. Don Schneider}
Prof. Donald Schneider has been involved with the Sloan Digital Sky
Survey since its earliest design stages in the 1980s and has
considerable experience in preparing large datasets for community use,
via leading several editions of the SDSS Quasar Catalogs and
participating in the annual public Data Releases. Prof. Schneider will
be on the follow-up Observational team, obtaining time on the HET if
necessary.


\subsection*{Dr. John Stott}
Dr John Stott is an expert in resolved NIR observations as a key
researcher in the KROSS VLT KMOS IFU survey of 800 z\gtrsim1
star-forming galaxies. He will use this expertise to explore and
analyse any spatially resolved spectral components of the MRS IFU
spectra. The goal is to spatially isolate regions of the galaxy to
hunt for those dominated by star formation or AGN.


\subsection*{Prof. Michael  Strauss}


\subsection*{Dr. Renske Smit}		
Dr. Smit is expert on IFU spectroscopy, both on {\it HST} and the VLT,
and will be contributing to the data analysis.

\subsection*{Prof. Nadia Zakamsaka} 


