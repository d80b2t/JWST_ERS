\section*{Summary of DD-ERS Data Products}
%Examples
Our primary SEP task and goal is to produce a Python package that
quickly manipulates and analyzes the full MRS Level 3 data, in
particular the MRS Spectral Cubes and 1D spectra.  We will take the
output from the third stage of the pipeline for MRS spectroscopy, the
CALIFU3 level data and analyse this.  We note there is already Python
legacy code for this type of analysis: {\href
https://spectral-cube.readthedocs.io/}{\tt
spectral-cube.readthedocs.io}.

\smallskip \smallskip
\noindent
With {\it a maximum} of 6 months between the first ERS observations
and the Cycle 2 GO Call for Proposals, 
this will likely be too short for full dissemination of our findings, novel
techniques and science results in the traditional manner, i.e. via
published journal articles. Moreover, ongoing updated versions of our
analyses and codes are envisaged to happen until right up to the Cycle 2
deadline.  To solve these issues, we will fully employ the power of a
code version repository system, in our case GitHub, to keep the
community informed and updated with our SEPs. GitHub {\it has code
versioning automatically built-in} so proper referencing of
e.g. technical notes is straight-forward.

%% 8.225 kpc/".
%% 3.5 × 3.5 arcsec field

\section*{Code Repositories and Documentation}
\vspace{-8pt}
We are already ensuring open access to representative data sets,
processeing pipelines and analysis tools in support of the preparation
of both Cycle 1 and Cycle 2 proposals.  The key links are:: 

\vspace{4pt}
\noindent
\href{https://github.com/d80b2t/JWST\_ERS}{{\tt github.com/d80b2t/JWST\_ERS}} (the P.I.'s public GitHub repository).
\vspace{4pt}

\noindent
\href{https://github.com/miri-mrs}{{\tt github.com/miri-mrs}} (a ``Organizational'' public repository for the Community).
\vspace{4pt}

\noindent
\href{http://miri-mrs.readthedocs.io/en/latest/}{{\tt miri-mrs.readthedocs.io}} (the global documentation)\\


\section*{MIRISim}
\vspace{-6pt}
MIRI data simulations (at ESA) include an Integral Field observation
with the Medium Resolution Spectrograph (MRS), a Low Resolution
Spectrograph (LRS) observation, and an imaging observation. (Credit:
ESA, Pamela Klaassen and the MIRISim Team).
%%
{\bf MIRISim does not produce final, calibrated data. 
MIRISim produces stuff that looks like it has come off of MIRI.}



\iffalse
\section{Useful links}
http://astroconda.readthedocs.io/en/latest/ \\
https://www.cosmos.esa.int/web/jwst/simulations\\
https://confluence.stsci.edu/display/JWSTDADF/JWST+Data+Analysis+Development+Forum\\
https://jwst.stsci.edu/science-planning/data-analysis-toolbox\\
https://www.youtube.com/watch?v=A024z9CITZs\\
https://jwst.stsci.edu/science-planning/proposal-planning-toolbox/simulated-data\\
\fi


\section*{Delivery Schedule for Science-Enabling Products.} 
\vspace{-6pt}
Our ERQ ERS proposal is the first part of a multi-cycle proposal
project and plan.  As such, we are {\it already highly motivated to
produce the data processing tools, codes, documentation and identify
the critical science-enabling products well in advance of the release
of the Cycle 2 Call for Proposals (September 2019).}
Our major milestones are::

\smallskip
\smallskip
\noindent
{\bf Before Cycle 1 GO Proposal Deadline (end Feb 2018):} 
Delivery of the first set (‘beta’) of MIRI MRS SEPs before the Cycle 1 GO Deadline (March 2018). 
Proin non tempus velit. Etiam laoreet, enim nec scelerisque dictum, tortor massa tempor enim, id pretium justo quam ac lectus. Maecenas diam nibh, interdum at lobortis sit amet, dignissim et quam. 

\smallskip
\smallskip
\noindent
 {\bf Prior to Launch (March 2018 - October 2018):} 
The release of our v1.0.0, with MIRSim mock data. 
Sed tincidunt faucibus risus, congue tempus nisl consectetur eget. Suspendisse venenatis turpis ut risus aliquam interdum. In at velit sed ligula dictum dignissim ut et dui. Curabitur ac scelerisque purus.

\smallskip
\smallskip
\noindent
{\bf Commissioning (November 2018-April 2019):} 
Ramp-up of the Postdoctoral Fellows. 
Donec elit tortor, scelerisque ac molestie id, hendrerit sit amet
ipsum. Maecenas non tempus sem. Pellentesque ut enim velit, eu
sagittis elit. Nulla in elementum erat.

\smallskip
\smallskip
\noindent
{\bf ERS/Cycle 1 (April 2019-September 2019):} 
Rapid version updates once the start of science operations commences
in April 2019.  In dictum arcu at nisi porttitor commodo. Donec felis
felis, elementum sit amet ultrices ac, interdum nec ante. Nullam eget
faucibus lectus. Donec vitae eros sapien, et faucibus ligula. Aenean
pharetra viverra fermentum.

\smallskip
\smallskip
\noindent
{\bf After Cycle 2 GO Deadline (post-September 2019):} 
Continue to provide regular (every 6 months) major version update.

\section*{Co-Investigators and Delivery of Science-Enabling Products.}
\vspace{-6pt}
Everyone named on this proposal is part of the ``core team''.  A
bibliography and SEP tasks of our team can be found
\href{https://github.com/d80b2t/JWST_ERS/blob/master/Proposal/CoI_bios.tex}{here}
and the SEP Deliverables are
\href{https://github.com/d80b2t/JWST_ERS/tree/master/Deliverables}{\tt
here}.

%\subsection{Dr. Nicholas Ross}
P.I. Dr. Nic Ross is a deep believer in delivering science-enabling
products, including datasets, catalogs, analysis codes, plots,
algorithms and where possible computational resources to the wide
astronomical community.  As such, the call for delivering
science-enabling products by the release of the Cycle 2 Call for
Proposals (September 2019) is fully inline with his scientific
practice.

\smallskip \smallskip
\noindent
Ross has being developing and buidling up his GitHub Repositories over
the last year or so, \href{https://github.com/d80b2t}{\tt
github.com/d80b2t} and indeed now does all his analysis and paper
writing on GitHub.

\smallskip \smallskip
\noindent
Ross will devote a considerable amount of his personal research time
(and due to his STFC ERF has 100\% FTE for research) to leading the
developement and timely prodcution of the ERS ERQ science-enabling
products.


\subsection{Dr. David Rosario} 
Co-PI Dr. David Rosario is awesome and also loves to write code. ;-)


\subsection{Prof. David Alexander} 
Prof. Alexander is an expert in high-$z$ obsured AGN.  He will use his
considerable {\it Spitzer IRS} expereince to help test our MIRI MRS
data-analysis toolkit.


\subsection{Dr. Rachael Alexandroff} 
Dr. Alexandroff is an leading expert on the ERQ population.  She will
bring to bear her now considerable and recent data analysis (long-slit
optical, polarimerty, radio) data analysis experience to build our
MIRI MRS data-analysis toolkit.


\subsection{Dr. Richard Bielby}

\iffalse
\subsection{Prof. Beth Biller}
Prof. Biller is an expert in infrared coronagraphic observations. 
While we do not intend to use the MIRI coronagraphs in this proposal, 
longer term observations would potentially involve observing the ERQs
with the Lyot or 4QPM if this became appropriate and technically feasible. 
\fi


\subsection{Prof. Niel Brandt}


\subsection{Dr. Rob Crain}
Dr. Rob Crain is a Royal Society University Research Fellow and will 
lead the theoretical team. 


\subsection{Prof. Xiaohui Fan}
Prof. Fan is a leader in surveys of high-redshift quasars and
reionization. He has extensive experience in studying quasars and
their host galaxies with {\it HST} and {\it Spitzer}.


\subsection{Prof. Fred Hamann}


\subsection{Prof. Dale Kocevski}
Prof. Kocevski is...
We are also asking for the appriopriate level of 
post-doctoroal support for Prof. Kocevski. 

\subsection{Prof. Linhua Jiang}


\subsection{Dr. Stephanie LaMassa}
Dr. LaMassa is currently at the STScI and is already involved with the
documentation efforts there. As such, Dr. LaMassa will help with those
efforts, along with writing code and potentially leading follow-up
where approriate. She will also be a natural link to the direct
efforts of the Space Telsecop Science Institute.


\subsection{Dr. Chelsea MacLeod}


\subsection{Dr. Ian McGreer}


\subsection{Prof. Brice Menard}	


\subsection{Dr. James Mullaney}


\subsection{Prof. Adam Myers}
Prof. Myers is an expert on the statistical analysis of reddened,
obscured and optically luminous quasars. He has co-authored many
well-cited publications on targeting quasars, quasar clustering,
high-redshift and unusual quasars, and quasars in the time
domain. Prof. Myers has made follow-up observations of quasars, and
other objects, at telescopes on five continents. His work has been
funded multiple times by the NSF and NASA, including via space
telescope programs such as those for {\it Chandra} and {\it Spitzer}. He has
served on time allocation committees for GALEX and the {\it HST}. 

\smallskip \smallskip
\noindent
Prof. Myers has also worked extensively in large survey
collaborations, often in formal management roles. He is an Architect
of SDSS-III and SDSS-IV, was the quasar target selection lead for the
SDSS-IV/eBOSS survey, is the Level 3 Target Selection Manager for the
Dark Energy Spectroscopic Instrument (DESI) and is the documentation
and website lead for the Legacy Surveys
(http://legacysurvey.org). {\it Prof. Myers is a strong advocate for
transparent and reproducible science. For example, as part of his work
on DESI, he has contributed over 10,000 lines of code to publicly
visible github repositories.}


\subsection{Dr. Jessie Runnoe}
Dr. Runnoe is an expert on quasar central engines at radio through
X-ray wavelengths.  Drawing on her vast observational experience, she
will contribute to the development of the MIRI MRS data-analysis
toolkit and assist with follow-up observations of the ERQ core sample.
She will be part of the Core Coding and Observational Follow-up
groups.

\subsection{Prof. Don Schneider}
Prof. Donald Schneider has been involved with the Sloan Digital Sky
Survey since its earliest design stages in the 1980s and has
considerable experience in preparing large datasets for community use,
via leading several editions of the SDSS Quasar Catalogs and
participating in the annual public Data Releases. Prof. Schneider will
be on the follow-up Observational team, obtaining time on the HET if
necessary.


\subsection{Prof. Tom Shanks}	


\subsection{Dr. John Stott}


\subsection{Prof. Michael  Strauss}


\subsection{Dr. Renske Smit}		


\subsection{Prof. Martin Ward}		


\subsection{Prof. Gillian Wright}


\subsection{Prof. Nadia Zakamsaka} 



