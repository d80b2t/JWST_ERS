We propose the perfect Director's Discretionary Early Release Science
(DD ERS) program. We have identified a population of mid-infrared
bright quasars at the peak of cosmological quasar activity. Their
global star-formation properties are currently unknown, but
observations with JWST MIRI, and in particular MRS spectroscopy, will
quantify the level of star-formation in these objects.
%%
Moreover, we have already begun to design and create science-enabling
products (SEPs) to help the community understand JWST's capabilities,
Our \href{https://github.com/miri-mrs}{{\tt MIRI MRS Code Repo}} is
already active and completely accessible to anyone in the broader
community.  We will deliver the MIRI MRS SEPs first with mock data
before the launch of JWST, and then in rapid fashion once the start of
science operations commences in April 2019.

\smallskip \smallskip
\noindent
Our teams commitment to an open access ideology, not only for data,
but for analysis codes, documentation, and scientific manuscripts is
already evident and in place (see also
\href{https://github.com/d80b2t}{{\tt the P.I.'s GitHub}}).  We aim to
engage a broad cross-section of the astronomical community, in
particular the extra-galactic community, in familiarizing themselves
with JWST data and scientific capabilities.

\smallskip \smallskip
\noindent
In this proposal we first outline the Scientific and Community Access
rational.  We give details of the scientific motivation in the
Scientific Justification and why our quasars are the ideal ERS
targets.  Details of our observations are given in the Technical
Description; in brief we will observe four quasars with MIRI MRS
across the full wavelength range. The exposure time for each object is
$xxxx$ seconds, and our entire program (with APT Smart Accounting
invoked) is 25.00 hours.  We give a list of Alternative Targets and we
do not have any special observational requirements, nor any Coordinated
Parallel Observations, and no duplications.  We give more details of
our plan to deliver the SEPs in the Data Processing \& Analysis Plan
section. This includes a list of analysis code modules we will
develop, how the community can rapidly access our findings and
technical notes, and the breakdown of who in the Core Analysis Team is
going to do what.



\section*{Science Rationale}
\noindent
Over 50 years after their formal identification, and over two decades
since the calculation of their space density evolution, several
fundamental facts remain unknown for high-luminosity AGN,
i.e. quasars: What is the main AGN triggering mechanism at the height
of quasar activity at redshifts $z=2-3$? What direct,
observational evidence in individual objects links AGN activity
to star formation?  Can we observe ``AGN feedback'' in action, in situ,  
for the most luminous sources at their peak activity? These remain the
outstanding observational extragalactic questions of our time. And
they will be answered with the launch of the {\it James Webb Space
Telescope}.

\smallskip \smallskip
\noindent
The recently identified Extremely Red Quasar (ERQ) objects are a
unique obscured quasar population with extreme physical conditions
related to powerful outflows across the line-forming regions. These
objects are found at the same cosmological epoch as the peak of quasar
activity, $z\approx2.5$, and are the best candidates to date to show
outflows on galactic scales in high-luminosity objects at these epochs; 
we are seeing quasar-level AGN feedback in action, in situ.

\smallskip \smallskip
\noindent
With our ERS MIRI observations of these ERQ, so called since they are
bright in the WISE W4 23$\mu$m band, we will deliver all the tools
necessary to the community in order to optimize Cycle 2 proposals of
5-30$\mu$m milliJansky bright sources. This is {\it a fundamental
tool} for the exploitation of a key MIRI instrument mode.

\smallskip \smallskip
\noindent
{\bf \underline{The Medium-Resolution Spectrometer}:}
The JWST MIRI medium-resolution spectrometer (MRS; Wells et al. 2015)
will observe simultaneous spatial and spectral information between 4.9
and 28.8 $\mu$m over a contiguous field of view up to 7.2" × 7.9" in
size. This is the only JWST configuration offering medium-resolution
spectroscopy (with $R\approx$1500-3500) longward of 5.2 $\mu$m.\\ MRS
observations are carried out using a set of 4 integral field units
(IFUs), each of which covers a different portion of the MIRI
wavelength range. MRS IFUs split the field of view into spatial
slices, each of which produces a separate dispersed ``long-slit"
spectrum. Post-processing produces a composite 3-dimensional (2
spatial and one spectral dimension) data cube combining the
information from each of these spatial slices. {\it This IFU aspect of
the Medium Resolution Spectrometer will allow, for the first time,
detailed investigations of the both the central AGN IR emission and
potentially extended emission.}



\section*{Community Access Rationale}
\noindent
We will satisfy the Goals and Principles of the DD ERS program by:
\begin{itemize}
\item ensuring open access to our datasets in support of the preparation of Cycle 2 proposals, and
\item engaging a broad cross-section of the astronomical community, in particular the extragalactic community, in familiarizing themselves with JWST data and scientific capabilities.
\end{itemize}

\noindent
In particular, we aim to produce several key science products:
\begin{itemize}
\item {\tt mrs\_analyzer} A Python module for analyzing MRS data; 
\item {\tt mrsfringe} A Python module for mitigating MRS fringing issues; 
\item a suite of science results on quasar feedback and evolution. 
\end{itemize}

\noindent
{\it Critically, we have already begun working closely with the MIRI team (due to the P.I.'s location at Edinburgh) and will continue to develop tools here for the MIRI Imager and MRS.}\\

\noindent
We will produce of SEP analysis code and documentation.  However, the 7
month period between the end of commissioning and Cycle 2 proposal
deadlines will be too short for dissemination of our findings, novel
techniques and potentially even science results in the traditional
manner (via journals). Moreover, ongoing updated versions 
of our analyses are envisaged to happen until right up to the Cycle 2 deadline. 
%%
To solve these issues, we will fully employ the power of a code version
repository system, in our case GitHub, to keep the community informed
and updated with or SEPs. GitHu {\it has code versioning automatically
built-in} so proper referencing of e.g. technical notes can easily






