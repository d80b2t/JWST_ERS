With our ERS MIRI observations of bright WISE W4 quasars, we will deliver all the tools necessary to the community in order to optimize Cycle 2 proposals of 5-30$\mu$m milliJansky bright sources. This is {\it a fundamental tool} for the exploitation of a key MIRI instrument mode

We will satisfy the Goals and Principles of the DD ERS program by:
\begin{itemize}
\item ensure open access to representative datasets in support of the preparation of Cycle 2 proposals, and
\item engage a broad cross-section of the astronomical community in familiarizing themselves with JWST data and scientific capabilities.
\end{itemize}


These goals distinguish the DD ERS program from standard GO investigations. In service of these goals, DD ERS proposals are invited from the community.
The DD ERS program is guided by the following key principles:

\begin{itemize}
\item Projects must be substantive science demonstration programs that utilize key instrument modes to provide representative scientific datasets of broad interest to researchers in major astrophysical sub-disciplines. Note that a meritorious DD ERS project need not cover every mode of the observatory. The request should match the focused science goals of the proposal.

\item Projects must design, create, and deliver science-enabling products to help the community understand JWST's capabilities.  An initial set of products must be delivered by the release of the Cycle 2 GO Call for Proposals (September 2019).  Each project must define a core team to be responsible for the timely delivery of such products according to a proposed project management plan, with performance subject to periodic review.

\item All observations must be schedulable within the first 5 months of Cycle 1 (planned to be from April to August 2019), and a substantive subset of the observations must be schedulable within the first three months.   Target lists must be flexible to accommodate possible changes to the scheduled start of science observations.

\item Both raw and pipeline-processed data will enter the public domain immediately after processing and validation at STScI. These data will have no exclusive access periods (i.e., no proprietary time).
\end{itemize}

\smallskip \smallskip
\noindent
STScI recognizes and supports the benefits of having diverse and inclusive scientific teams involved in the formulation of ERS proposals.  Programs with diverse representation of community members in a given sub-discipline helps ensure that the investigations will be of broad interest. Broad involvement also facilitates the dissemination of JWST expertise through a more extensive network, and promotes more equitable participation in JWST scientific discovery.

\smallskip \smallskip
\noindent
The DD ERS program will be essential for informing the scientific and technical preparation of Cycle 2 General Observer (GO) proposals, submitted seven months after the end of commissioning.

{\it Critically, we have alreayd begun working closely with the MIRI team (due to the P.I.'s location at Edinburgh) and will continue to develop tools here for the MIRI Imager and MRS.}



\medskip
\medskip
\smallskip
\smallskip
\noindent
{\bf \underline{Relation to Spitzer IRS}:}\\

Major Achievement of Spitzer was IRS. \\
However IRS had fringing. \\
Also, Spitzer IRS died before WISE; therefore now WISE W4 objects were observed by the IRS.\\


