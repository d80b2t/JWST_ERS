\vspace{-4pt}
\section*{{\sc Executive Summary}}

\vspace{-6pt}
\noindent
{\it 1. Justification for ERS time}\\
Our Director’s Discretionary time for an Early Release Science
(DD-ERS) program will help the community learn about the longest
wavelength spectrograph on the {\it James Webb Space Telescope (JWST)},
the Mid-Infrared Instrument (MIRI) Medium-resolution spectrometer
(MRS).  We will release a suite of science-enabling products (SEPs)
via a public data analysis and code repository that we have already
begun to build: \href{https://github.com/miri-mrs}{\tt
github.com/miri-mrs}.  The accompanying documentation is also already
being written: \href{http://miri-mrs.readthedocs.io/}{{\tt
miri-mrs.readthedocs.io}}.  {\it Our primary SEP goal is to produce a
Python package that quickly manipulates and analyzes the full MRS
Level 3 data, in particular the MRS Spectral Cubes and 1D spectra. }

\smallskip 
\smallskip \smallskip
\noindent
{\it 2. Project Management Plan \& Budget} \\
Our team members have a long and proven history of delivering SEPs, 
catalogs, data products, web access pages, documentation and the
necessary helpdesk support for large collaborations on strict
deadlines.  Our project is led by a STFC Ernest Rutherford Senior
Research Fellow, who is able to contribute 100\% FTE to the
development, delivery and management of the SEPs.  We also ask for
support for two postdoctoral researchers who will be in place at
Launch time.

\smallskip 
\smallskip \smallskip
\noindent
{\it 3. Scientific Justification}\\
Our science case is straight-forward, yet strikes at the heart of a
major and still open extragalactic astrophysical question: {\it What
are the star-formation properties of mid-infrared luminous quasars at
the peak of quasar activity? } We will answer this by looking for the
presence of polycyclic aromatic hydrocarbon (PAH) spectral features in
$z\approx2.5$ infrared bright quasars.  Furthermore, we will use
the IFU capability of MIRI MRS in order to quantify the spatial
location of the IR luminosity. This is an ideal investigation for
{\it James Webb}; no other current or near-future facility, ground or space-based, has the combination of MIR
spectroscopy, angular resolution and the sensitivity required for
accessing the PAH spectral features at $z>2$, {\it and} being able to
spatial resolve their structure.


\smallskip 
\smallskip \smallskip
\noindent
{\it 4. Description of the Targets}\\
We have four primary targets; all are available for early observation.
We also have a back-up list of ten secondary targets, any of which would 
allow us to achieve our SEP and Science goals. These fourteen objects have 
extensive associated 
multiwavelength data with our four primary targets known 
to exhibit interesting kinematic behavior. 


\smallskip 
\smallskip \smallskip
\noindent
{\it 5. Team Diversity}\\
Our team is an ensemble of observational extragalactic experts with a
broad geographical dispersion. This is a new collaboration, but with
substantial heritage and expertise from the SDSS, the {\it
HST/Chandra} Deep Field surveys, and more recent ground-based IR IFU
collaborations (e.g., VLT/KMOS).  Our team currently has a gender binary split
of 7:12 (37:63) F:M. It is predominantly White European or White American.
%% Alexander    Alexandroff
%% Brandt          Banerji
%% Fan               MacLeod
%% Hamann       Runnoe
%% Kocevski      Smit
%% Mullaney      Zakamska
%% Myers           LaMassa
%% Rosario
%% Ross
%% Schneider
%% Stott
%% Strauss


\section*{Science Rationale}
\vspace{-6pt}
\noindent
Over 50 years after their formal identification, and over two decades
since the calculation of their space density evolution, several
fundamental facts remain unknown for high-luminosity AGN,
i.e. quasars: What is the main AGN triggering mechanism at the height
of quasar activity at redshifts $z=2-3$? What direct observational
evidence in individual objects links AGN activity to star formation?
Can we observe ``AGN feedback'' in action, in situ, for the most
luminous sources at their peak activity?  Such unknowns about the
co-evolution of black holes and their host galaxies remain among the
most fundamental unanswered questions in extragalactic astronomy.  And
they will be answered with the launch of the {\it James Webb Space
Telescope}.

\smallskip \smallskip
\noindent
We have identified a population of obscured, mid-infrared bright
quasars at the peak of cosmological quasar activity.  These sources
are mid-IR luminous and may be powered by major bursts of star
formation tied to an early phase of galaxy
evolution/formation. However, their global star-formation properties
are currently unknown.  Observations with {\it JWST} MIRI, and in
particular MRS spectroscopy, will quantify the level of star-formation
in these objects.  {\it We will observe four ``extremely red'' quasars
with MIRI MRS across the full wavelength range to high signal to
noise.} These observations will address the fundamental question of
the link between star-formation and AGN activity, by quantifying these
two quantities and studying the morphological and kinematic properties
for both of these processess; an investigation {\it JWST} was
specifically built for.

\smallskip \smallskip
\noindent
It is unknown whether the large IR luminosities observed in these
quasars is from star formation, which would produce strong polycyclic
aromatic hydrocarbon (PAH) spectral features, or, if it is from the hot
dust near the central quasar, which should produce much weaker/no PAH
emission (due to the AGN MIR emission diluting and even destroying PAH
features). {\it Via the detection, or otherwise, of PAH spectral
features, we will measure the SFRs during what is potentially a very
early/obscured stage of massive galaxy formation in the extremely red
quasar population.}

\smallskip \smallskip
\noindent
MIRI MRS is the instrument of choice since no other spectrometer on
{\it JWST} observes longward of 5.3$\mu$m; going redder than this is
crucial in order to detect PAH features in $z>2$ objects.  If present
we will observe the most prominent, well-known major PAH emission
features at 3.3, 6.2, 7.7, and 8.6$\mu$m. The mid-IR spectral region
also presents a suite of high-ionization lines
and critically,  we will have access to the \nevi 7.65$\mu$m line which 
can be used to measure the instantaneous luminosity of the central engine.
 {\it The desire to immediately
gain high signal-to-noise spectra in order to investigate the physics
and chemistry of quasar PAHs, along with observational overhead
concerns, pushes us to observe each object for 3.6 hours, for a total
program Charged Time of 22.20 hours.}

\section*{Community Access Rationale}
\vspace{-6pt}
\noindent
We have already begun to design and create science-enabling products
(SEPs) to help the community understand JWST's capabilities.  Our MIRI
MRS Repo \href{https://github.com/miri-mrs}{{\tt github.com/miri-mrs}}
is active and completely accessible to anyone in the broader
community.  The accompanying documentation is also already being
written: \href{http://miri-mrs.readthedocs.io/}{{\tt
miri-mrs.readthedocs.io}}.  {\it Our primary SEP goal is to produce a
Python package that quickly manipulates and analyzes the full MRS
Level 3 data, in particular the MRS Spectral Cubes and 1D spectra. }
We note there is already Python legacy code for this type of analysis:
\href{https://spectral-cube.readthedocs.io/}{\tt
spectral-cube.readthedocs.io}. \noindent {\it Critically, we have
already begun working closely with the MIRI team (due to the P.I.'s
location at Edinburgh) and will continue to develop tools here for the
MIRI MRS.}

\smallskip \smallskip
\noindent
Our timeline has delivery of the first set (`beta') of MIRI MRS SEPs
before the Cycle 1 GO Deadline (March 2018); our v1.0.0 (with
e.g. MIRSim mock data) before the launch of JWST (October 2018) and
then rapid version updates once the start of science operations
commences in April 2019.

\smallskip \smallskip
\noindent
With {\it a maximum} of 6 months between the first ERS observations
and the Cycle 2 GO Call for Proposals, 
this will likely be too short for full dissemination of our findings, novel
techniques and science results in the traditional manner, i.e. via
published journal articles. Moreover, ongoing updated versions of our
analyses and codes are envisaged to happen until right up to the Cycle 2
deadline.  To solve these issues, we will fully employ the power of a
code version repository system, in our case GitHub, to keep the
community informed and updated with our SEPs. GitHub {\it has code
versioning automatically built-in} so proper referencing of
e.g. technical notes is straight-forward.

\smallskip \smallskip
\noindent
Our team's commitment to an open access ideology, not only for data,
but for analysis codes, documentation, and scientific manuscripts is
already evident and in place, for an example, see the P.I.'s GitHub
\href{https://github.com/d80b2t}{{\tt /github.com/d80b2t}}.  We are
thus extremely well-placed to satisfy the overall goals of the DD ERS
program.  ensure open access to representative datasets in support of
the preparation of Cycle 2 proposals, and engage a broad cross-section
of the astronomical community in familiarizing themselves with JWST
data and scientific capabilities.

\medskip \medskip
\medskip \medskip

