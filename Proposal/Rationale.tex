
DD-ERS proposal requirements (not your standard GO proposal…)

1. Justification for ERS time
    How will your program help the community learn to do science with JWST
    and prepare for Cy2?  What science-enabling products are proposed for
    development and release? How will your program demonstrate baseline JW
    science capabilities?
    
2. Project Management Plan \& Budget
    Data processing and analysis plan, roles, responsibilities, work
    schedule, budget. Identification of core team responsible for
    science-enabling product development and delivery.

3. Scientific Justification
    Why are the observations scientifically compelling and require JWST?.

4. Description of the Observations
    Establish feasibility for early execution, and flexibility in
    target selection to accommodate any change to start date for Cy1
    science obs.

5. Team Diversity
    Description of how the proposing team represents and has input
    from diversity of experts with broad demographics within
    sub-discipline
	

DD-ERS Data Products

Examples

1.	Participation in community briefings organized by STScI
written/oral updates on progress
written/oral documentation of lessons-learned
2.	Software with documentation
data analysis and modeling
enhanced data processing
3.	Enhanced data products with documentation
aligned multiband images, extracted spectra, etc
catalogs

Building blocks for GO programs 
APT files available at Cy1 CP release
Enables Cy1 archival programs

Provides kernels for cohesive science-based training sets 
Observation & proposal planning
Data reduction and analysis cookbooks










We have identified a population of obscured, mid-infrared bright
quasars at the peak of cosmological quasar activity.  These sources are
mid-IR luminous and may be powered by major bursts of star formation
tied to an early phase of galaxy evolution/formation. However, their
global star-formation properties are currently unknown, but
observations with JWST MIRI, and in particular MRS spectroscopy, will
quantify the level of star-formation in these objects.  {\it We will
observe four ``extremely red'' quasars with MIRI MRS across the full wavelength range to
very high signal to noise.} These observations will address the
fundamental question of the link between star-formation and AGN
activity, at the cosmological epoch for both of these processess; an
investigation JWST was specifically built for.

\smallskip \smallskip
\noindent
Moreover, we have already begun to design and create science-enabling
products (SEPs) to help the community understand JWST's capabilities,
Our \href{https://github.com/miri-mrs}{{\tt MIRI MRS Code Repo}} is
already active and completely accessible to anyone in the broader
community.  We will deliver the MIRI MRS SEPs first with mock data
before the launch of JWST, and then in rapid fashion once the start of
science operations commences in April 2019.
%%
Our team's commitment to an open access ideology, not only for data,
but for analysis codes, documentation, and scientific manuscripts is
already evident and in place (for an example, see
\href{https://github.com/d80b2t}{{\tt the P.I.'s GitHub}}).  Via these
code repos, and extensive documentation efforts, we aim to engage a
broad cross-section of the astronomical community, in particular the
extra-galactic community, in familiarizing themselves with JWST data
and its scientific capabilities.


\section*{Science Rationale}
\vspace{-6pt}
\noindent
Over 50 years after their formal identification, and over two decades
since the calculation of their space density evolution, several
fundamental facts remain unknown for high-luminosity AGN,
i.e. quasars: What is the main AGN triggering mechanism at the height
of quasar activity at redshifts $z=2-3$? What direct,
observational evidence in individual objects links AGN activity
to star formation?  Can we observe ``AGN feedback'' in action, in situ,  
for the most luminous sources at their peak activity? These remain the
outstanding observational extragalactic questions of our time. And
they will be answered with the launch of the {\it James Webb Space
Telescope}.

\smallskip \smallskip
\noindent
The recently identified Extremely Red Quasar (ERQ) objects are a
unique obscured quasar population with extreme physical conditions
related to powerful outflows across the line-forming regions. These
objects are found at the same cosmological epoch as the peak of quasar
activity, $z\approx2.5$, and are the best candidates to date to show
outflows on galactic scales in high-luminosity objects at these epochs. 

\smallskip \smallskip
\noindent
However, it is unknown whether the large IR luminosities observed in 
these quasars is from star formation, which should produce strong 
polycyclic aromatic hydrocarbon (PAH) spectral features, 
or if the hot dust near the central quasar, which should produce no PAH, 
is driving the MIR luminosity. 
{\it Via the detection, or otherwise, of PAH spectral features, we will 
measure the SFRs during what is potentially a very early/obscured stage 
of massive galaxy formation in the Extremely Red Quasar population.}

\smallskip \smallskip
\noindent
MIRI MRS is the instrument of choice since no other medium resolution
spectrometer on JWST observes longward of 5$\mu$m; going redder than
this is crucial in order to detect PAH features in $z>2$ objects.  The
desire to immediately gain high signal-to-noise spectra in order to
investigate the physics and chemistry of quasar PAHs, along with
observational overhead concerns, pushes us to observe each object for
3.6hours, for a total program Charged Time of 22.20 hours.



\section*{Community Access Rationale}
\vspace{-6pt}
\noindent
We will satisfy the Goals and Principles of the DD ERS program of
ensuring open access to our datasets and analyses in support of the
preparation of Cycle 2 proposals, and, engaging a broad cross-section
of the astronomical community, in particular the extragalactic
community, in familiarizing themselves with JWST data and scientific
capabilities by utilizing modern analysis (code writing) repository
tools and documentation that is standardized and publicly editable.

\noindent
In particular, we aim to produce several key science products:
\begin{itemize}
\item {\tt mrs\_analyzer} A Python module for analyzing MRS data; 
\item {\tt mrsfringe} A Python module for mitigating MRS fringing issues; 
\item a suite of science results on quasar feedback and evolution. 
\end{itemize}

\noindent
{\it Critically, we have already begun working closely with the MIRI
team (due to the P.I.'s location at Edinburgh) and will continue to
develop tools here for the MIRI MRS.}
%%
We will produce of SEP analysis code and documentation.  Critically,
with {\it a maximum} of 6 months between the first ERS observations
(April 2019) and the Cycle 2 GO Call for Proposals (September 2019),
this will be too short for dissemination of our findings, novel
techniques and science results in the traditional manner, i.e. via
published journal articles. Moreover, ongoing updated versions of our
analyses are envisaged to happen until right up to the Cycle 2
deadline.  To solve these issues, we will fully employ the power of a
code version repository system, in our case GitHub, to keep the
community informed and updated with or SEPs. GitHu {\it has code
versioning automatically built-in} so proper referencing of
e.g. technical notes is straight-forward.






