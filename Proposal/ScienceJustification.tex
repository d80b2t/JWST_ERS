\newcommand{\imw}{$i$--$W3$}
\newcommand{\imwf}{$i$--$W4$}
\newcommand{\rmwf}{$r$--$W4$}
\newcommand{\imwt}{$i$--$W2$}
\newcommand{\wtmwf}{$W3$--$W4$}
%\newcommand{\kms}{km s$^{-1}$}
\newcommand{\cmN}{cm$^{-2}$}
\newcommand{\cmn}{cm$^{-3}$}
\newcommand{\msun}{M$_{\odot}$}
\newcommand{\lsun}{L$_{\odot}$}
\newcommand{\lam}{$\lambda$}
\newcommand{\mum}{$\mu$m}
\newcommand{\ebv}{$E(B$$-$$V)$}
\newcommand{\heii}{\mbox{He\,{\sc ii}}}
\newcommand{\cv}{\mbox{C\,{\sc v}}}
\newcommand{\civ}{\mbox{C\,{\sc iv}}}
\newcommand{\ciii}{\mbox{C\,{\sc iii}}}
\newcommand{\cii}{\mbox{C\,{\sc ii}}}
\newcommand{\nv}{\mbox{N\,{\sc v}}}
\newcommand{\niv}{\mbox{N\,{\sc iv}}}
\newcommand{\niii}{\mbox{N\,{\sc iii}}}
\newcommand{\oi}{\mbox{O\,{\sc i}}}
\newcommand{\oii}{\mbox{O\,{\sc ii}}}
\newcommand{\oiii}{\mbox{[O\,{\sc iii}]}}
\newcommand{\oiv}{\mbox{O\,{\sc iv}}}
\newcommand{\ov}{\mbox{O\,{\sc v}}}
\newcommand{\ovi}{\mbox{O\,{\sc vi}}}
\newcommand{\ovii}{\mbox{O\,{\sc vii}}}

\newcommand{\feii}{\mbox{Fe\,{\sc ii}}}
\newcommand{\feiii}{\mbox{Fe\,{\sc iii}}}
\newcommand{\mgii}{\mbox{Mg\,{\sc ii}}}
\newcommand{\neii}{[Ne\,{\sc ii}]\ }
\newcommand{\neiii}{[Ne\,{\sc ii}]\ }
\newcommand{\nev}{Ne\,{\sc v}\ }
\newcommand{\nevi}{[Ne\,{\sc vi}]\ }
\newcommand{\neviii}{\mbox{Ne\,{\sc viii}}}
\newcommand{\aliii}{\mbox{Al\,{\sc iii}}}
\newcommand{\siii}{\mbox{Si\,{\sc ii}}}
\newcommand{\siiii}{\mbox{Si\,{\sc iii}}}
\newcommand{\siiv}{\mbox{Si\,{\sc iv}}}
%\newcommand{\lya}{\mbox{Ly$\alpha$}}
%\newcommand{\lyb}{\mbox{Ly$\beta$}}
\newcommand{\hi}{\mbox{H\,{\sc i}}}
\newcommand{\snine}{\mbox{[S\,{\sc ix}]}}
\newcommand{\sivi}{\mbox{[Si\,{\sc vi}]}}
\newcommand{\sivii}{\mbox[{Si\,{\sc vii}]}}
\newcommand{\siix}{\mbox{[Si\,{\sc ix}]}}
\newcommand{\six}{\mbox{[Si\,{\sc x}]}}
\newcommand{\sixi}{\mbox{[Si\,{\sc xi}]}}
\newcommand{\caviii}{\mbox{[Ca\,{\sc viii}]}}
\newcommand{\arii}{\mbox{[Ar\,{\sc ii}]}}

%%[Ar II] 6.97
%% [S IX] 1.252 μm 328 
% [Si X] 1.430 μm 351 
% [Si XI] 1.932 μm 401 
% [Si VI] 1.962 μm 167 
% [Ca VIII] 2.321 μm 128 
% [Si VII] 2.483 μm 205 
% [Si IX] 3.935 μm 303
% [Ar II] 6.97


%\snine\ at 1.252$\mu$m, \six\ at 1.430$\mu$m, \sixi\ at 1.932$\mu$m, \sivi\ at
%1.962$\mu$m, \caviii\ at 2.321$\mu$m, \sivi\ at 2.483$\mu$m \siix\ at
%3.935$\mu$m and \arii\ at 6.97$\mu$m. 
%%
%% such as [Ne ii]12.8 μm, [Ne v]14.3 μm, [Ne iii]15.5 μm, [S iii]18.7 μm and 33.48 μm, [O iv]25.89 μm and [Si ii]34.8 μm (e.g
%%
%% MIR emission lines like [NeII] and [NeV] are ..
%%
%% Also,  arXiv:astro-ph/0003457v1 
%% [NeV] 14.32um & 24.32um and [NeVI] 7.65um imply an A(V)>160 towards the NLR...
%% [NeIII]15.56um/[NeII]12.81um
%%
%% [Ne V] 14.3, 24.2 μm 97.
%% [Ne II] 12.8 μm
%% [OIV] 26μm
%%


One lasting scientific legacy of the {\it Hubble Space Telescope
(HST)} will be the discovery of giant black holes at the centers of
galaxies, confirming the longstanding theory of the ``central
engines'' of quasars. One of the major surprises from the {\it Hubble}
was the discovery of a correlation between black hole mass and galaxy
properties.  This connection, causal or otherwise may provide crucial
clues to how and why these black holes formed and how their host
galaxies evolved. As of the launch of the {\it James Webb Space
Telescope (JWST), this is one of the outstanding questions in
astrophysics.}
%%Assessment of Options for Extending the Life of the Hubble Space Telescope: Final Report (2005)
%% https://www.nap.edu/read/11169/chapter/5

Furthermore, discovery of spectral lines in active galaxies reveals that black holes can trigger massive star formation. 
As such, %the two main energy sources available to a galaxy are nuclear fusion in stars and gravitational accretion onto compact objects. 
%%
The link between massive galaxies and the central super-massive black holes (SMBHs) that seem ubiquitous in them is now thought to be vital to the understanding of galaxy formation and evolution ([1], [2]).  As such, huge observational and theoretical effort has been invested in trying to measure and understand the physics involved in these enigmatic systems.\\

\smallskip
\smallskip
\noindent
Hubble (more or less) discovered $M-\sigma$. \\
Gives rise to the idea of AGN/QSO feedback (in order to shape the LF at the high-mass end)\\
However, {\it direct observational evidence} for AGN feedback is conspicuous by its absence. This is especially true at high-$z$, e.g. $z=2-3$, at the height of the Quasar Epoch. \\
We have identified the best candidates that suggest we are seeing quasar feedback in action, in situ at high-redshift. These are the ``Extremely Red Quasars'' identified via their WISE W3/4 colors. \\
As such, these milliJansky luminous AGN {\it are ideal targets for JWST MIRI}. 




\medskip
\medskip

\smallskip
\smallskip
\noindent
{\bf \underline{The Extremely Red Quasar Population}:}
By matching the quasar catalogues of the Sloan Digital Sky Survey
(SDSS), the Baryon Oscillation Spectroscopic Survey (BOSS) to the
Wide-Field Infrared Survey Explorer (WISE), Ross et al. (2015)
discovered quasars with extremely red infrared-to-optical colours:
$r_{\rm AB}-W4_{\rm Vega}>14$ mag, i.e., $F_\nu({\rm 22\mu
m})/F_\nu(r) \gtrsim 1000$.  These objects have infrared luminosities
$\sim 10^{47}$ erg s$^{-1}$.  The original motivation here was to look
for PAHs that had redshifted in the WISE W4 band for $z\approx$2.5
quasars in order to study and link luminous AGN activity and star
formation at the ``height of the quasar epoch''.

Hamann et al. (2017) then fully and properly refined the selection of
the ERQs, changed the definition based on other data and common
properties, and indeed found many more objects in this new scheme. The
ERQs have a suite of peculiar emission-line properties including large
rest equivalent widths (REWs), unusual ``wingless'' line profiles,
large \nv /\lya , \nv /\civ , \siiv /\civ\ and other flux ratios, and
very broad and blueshifted [\oiii ] \lam 5007.

In Hamann et al., our team identified a ``core'' sample of 97 ERQs
with nearly uniform peculiar properties selected via \imw\ $\ge 4.6$
(AB) and REW(\civ ) $\ge$ 100 \AA\ at redshifts 2.0--3.4. A broader
search finds 235 more red quasars with similar unusual
characteristics. The core ERQs have median luminosity $\left<\log L
({\rm ergs/s})\right> \sim 47.1$, sky density 0.010 deg$^{-2}$,
surprisingly flat/blue UV spectra given their red UV-to-mid-IR colors,
and common outflow signatures including BALs or BAL-like features and
large \civ\ emission-line blueshifts. Their SEDs and line properties
are inconsistent with normal quasars behind a dust reddening
screen. These ``Core ERQs'' are a unique obscured quasar population
with extreme physical conditions related to powerful outflows across
the line-forming regions. Patchy obscuration by small dusty clouds
could produce the observed UV extinctions without substantial UV
reddening.


\smallskip
\smallskip
\noindent
In Zakamska et al. (2016) we used XShooter/VLT to measure rest-frame
optical spectra of four $z\sim 2.5$ extremely red quasars. We
discovered very broad (full width at half max$= 2600-5000$ km
s$^{-1}$), strongly blue-shifted (by up to 1500 km s$^{-1}$)
\oiii$\lambda$5007\AA\ emission lines in these objects. In a large
sample of type 2 and red quasars, \oiii kinematics are positively
correlated with infrared luminosity, and the four objects in our
sample are on the extreme end both in \oiii kinematics and infrared
luminosity.
%%
As such, we estimate that at least 3\% of the bolometric luminosity in
these objects is being converted into the kinetic power of the
observed wind. Photo-ionization estimates suggest that the \oiii
emission might be extended on a few kpc scales, which would suggest
that the extreme outflow is affecting the entire host galaxy of the
quasar. These sources may be the signposts of the most extreme form of
quasar feedback at the peak epoch of galaxy formation, and may
represent an active ``blow-out'' phase of quasar evolution. 


\smallskip
\smallskip
\noindent
Alexandroff et al. (2017 submitted and in prep.).... \\




\medskip
\medskip
\smallskip
\smallskip
\noindent
{\bf \underline{Relation to Spitzer IRS}:}
Major Achievement of Spitzer was IRS. \\
$R\sim600$, now $R\sim2000s$, which allows {\it chemistry.}\\
Spitzer IRS died before WISE; therefore now WISE W4 objects were observed by the IRS.\\
i.e. no $z\sim2.5$ ERQs w/ feedback in action were observed. \\



\medskip
\medskip
\smallskip
\smallskip
\noindent
{\bf \underline{MIRI Imaging}:}
Imaging of the ERQs will tell us what environments they live in (currently totally unknown).\\
Just look at the ERQs: flux from central source $\Rightarrow$ AGN; flux from extended 
$\Rightarrow$ SF; Big puzzle since Spitzer (e.g. and cf. the submm population). \\



\medskip
\medskip
\smallskip
\smallskip
\noindent
{\bf \underline{Next Steps}:}
%I think the easiest useful thing we could get from NIR spectra is a test for any PAH emission at all. The mid-IR emission is booming in the ERQs. The PAHs would show that it's dominated by star formation. But my guess is that it’s dominated by hot dust in the torus, so no PAHs. It’s an interesting test either way, but to make it more appealing we should think about other lines available in the MIR. I don’t know anything about JWST capabilities, but if we can search for some forbidden lines used in local ULIRG/AGN studies, we could say a lot more about the kinematics and what powers the lines. 
% This paper by Veilleux seems like a good guide for lines that might be available: https://arxiv.org/abs/astro-ph/0201118
