

The lasting scientific legacy of the {\it Hubble Space Telescope} will
be the discovery of giant black holes at the centers of galaxies,
confirms longstanding theory of the “central engines” of quasars. One
of the major surprises from the {\it Hubble} was the discovery of a
correlation between black hole mass and galaxy properties.  This may
provide crucial clues to how and why these holes formed.
%%Assessment of Options for Extending the Life of the Hubble Space Telescope: Final Report (2005)
%% https://www.nap.edu/read/11169/chapter/5

Furthermore, discovery of spectral lines in active galaxies reveals that black holes can trigger massive star formation.

Giant Black Holes at the Centers of Galaxies
Hubble’s high angular resolution allows astronomers to peer into the hearts of galaxies to measure the orbital speeds of gas and stars close to their centers. The speeds of stars reach 1000 km/s in many objects, thereby indicating the presence of intense gravitational fields caused by massive black holes of up to a billion solar masses. Though mostly invisible today, these black holes shone brilliantly in the past as quasars, fueled by the infall of then-abundant interstellar gas. Key data found by the Hubble telescope reveal a correlation between black hole mass and galaxy properties that may provide crucial clues to how and why these holes formed.


The two main energy sources available to a galaxy are nuclear fusion in stars and gravitational accretion onto compact objects. 
%%
The link between massive galaxies and the central super-massive black holes (SMBHs) that seem ubiquitous in them is now thought to be vital to the understanding of galaxy formation and evolution ([1], [2]).  As such, huge observational and theoretical effort has been invested in trying to measure and understand the physics involved in these enigmatic systems.


Hubble discovered $M-\sigma$. \\
Gives rise to the idea of AGN/QSO feedback (in order to shape the LF at the high-mass end)\\
However, {\it direct observational evidence} for AGN feedback is conspicuous by its absence. This is especially true at high-$z$, e.g. $z=2-3$, at the height of the Quasar Epoch. \\
We have identified the best candidates that suggest we are seeing quasar feedback in action, in situ at high-redshift. These are the ``Extremely Red Quasars'' identified via their WISE W3/4 colors. \\
As such, these milliJansky luminous AGN {\it are ideal targets for JWST MIRI}. 
