%%%%%%%%%%%%%%%%%%%%%%%%%%%%%%%%%%%%%%%%%%%%%%
%%
%%    D e s c r i p t i o n    o f     t h e      O b s e r v a t i o n s  : 
%%
%%%%%%%%%%%%%%%%%%%%%%%%%%%%%%%%%%%%%%%%%%%%%%

\smallskip \smallskip
\noindent
We propose to use MIRI MRS to observe four quasars. Details of our 
four quasar targets are given in Table~\ref{tab:targets}. 

\smallskip \smallskip
\noindent
%% Describe the targets and observational modes to be used. Quantitative
%% estimates must be provided of the accuracy required to achieve key
%% science goals. 
%%
%% Proposers must demonstrate that all observations can
%% execute in the first 5 months of Cycle 1 (planned to be from April to
%% August 2019), and that a substantive subset of the observations are
%% accessible in the first 3 months. This description should also include
%% the following::


\begin{enumerate}[label=\alph*]
    \item{Plan for Alternative Targets: }
      
    \item{There are no Special Observational Requirements for our ERS ERQ.} 

    \item{We are not inducing Coordinated Parallels observations.}

    \item{The are no duplicated observations.} 
\end{enumerate}

\section{MIRI Overview}
The JWST Mid-Infrared Instrument (MIRI) provides imaging and
spectroscopic observing modes from 4.9 to 28.8 μm (Wright et al. 2015,
Rieke et al. 2015). MIRI offers a very broad range of observing modes,
including: imaging; low-resolution slitted and slitless spectroscopy;
medium-resolution integral field unit (IFU) spectroscopy and
coronagraphy.  We will be utilising the medium-resolution integral
field unit (IFU) and spectroscopy mode.
Medium-resolution spectroscopy observing mode has a wavelength 
coverage of 4.9–28.8$\mu$m, a Field of view from 3.9''x3.9'' (for Channel 1) 
to 7.7''x7.7'' (for Channel 4), a pixel scale of 0.196–0.273 ''/pixel and 
a resolving power of $R=\lambda / \Delta lambda = 1550-3250$. 
The FWHm is 2 pix at 6.2$\mu$m (where FWHM = 0.314" $\times (\lambda / 10 \mu$m) 
for $\lambda  > 8$\mum.) 


\section{MIRI Observing Modes}
%    \subsection{MIRI Imaging}
  %  \subsection{MIRI Coronagraphic Imaging}
    %\subsection{MIRI Low-Resolution Spectroscopy}
    \subsection{MIRI Medium-Resolution Spectroscopy}
    
\section{MIRI Instrumentation}
    \subsection{MIRI Optics and Focal Plane}
    \subsection{MIRI Filters and Dispersers}
    \subsection{MIRI Coronagraph Masks}
    \subsection{MIRI Spectroscopic Elements}
    \subsection{MIRI Detector Overview}
        \subsubsection{MIRI Detector Readout Overview}
          \begin{itemize}
            \item{MIRI Detector Readout Slow}
            \item{MIRI Detector Readout Fast}
          \end{itemize}
         \subsubsection{MIRI Detector Performance}
         \subsubsection{MIRI Detector Subarrays}


\section{MIRI Operations}
\subsection{MIRI Target Acquisition Overview}
\subsubsection{MIRI Coronagraphic Imaging Target Acquisition}
\subsubsection{MIRI LRS Target Acquisition}
\subsubsection{MIRI MRS Target Acquisition}
\subsubsection{MIRI Bright Source Imaging Target Acquisition}

\subsection{MIRI Dithering Overview}
\subsubsection{MIRI Imaging Dithering}
\subsubsection{MIRI Coronagraph Imaging Dithering}
\subsubsection{MIRI LRS Dithering}
\subsubsection{MIRI MRS Dithering}
\subsubsection{MIRI MRS Dedicated Sky Observations}

\subsection{MIRI Mosaics Overview}
\subsubsection{MIRI Imaging Mosaics}
\subsubsection{MIRI MRS Mosaics}

\subsection{MIRI Time Series Observations}
\subsubsection{MIRI LRS Time Series Observations}

\section{MIRI Predicted Performance}
\subsection{MIRI Bright Source Limits}
\subsection{MIRI Sensitivity}

\smallskip \smallskip
\noindent
Specifying JWST Position Angles, Ranges, and Offsets.\\
https://jwst-docs.stsci.edu/display/JPP/Specifying+JWST+Position+Angles\%2C+Ranges\%2C+and+Offsets\\
https://jwst-docs.stsci.edu/display/JPP/JWST+APT+Special+Requirements\\


\begin{table}
\begin{center}
\begin{tabular}{||  l|l|l|l|l ||}
  \hline\hline
  &&&& \\
  Object Name (SDSS)        & J0834+0159         &  J1232+0912          & J2215-0056        & J2323-0100 \\
  &&&& \\
  \hline
  &&&& \\
  Object R.A.                             & 08:34:48.48         & 12:32:41.73           & 22:15:24.00          & 23:23:26.17     \\
  object declination                  & $+$01:59:21.1     & $+$09:12:09.3      & $-$00:56:43.8      & $-$01:00:33.1  \\
  $r$-band AB magnitude         & 21.20$\pm$0.05  & 21.11$\pm$ 0.05  & 22.27$\pm$0.12  & 21.62$\pm$ 0.08 \\  
  WISE W4-band Vega magnitude & 6.88$\pm$0.09  & 6.78 $\pm$0.09   & 7.91$\pm$0.24  & 7.76$\pm$0.22 \\  
  WISE W4-band flux, $F_{\nu}$   & $>$6 mJy             & $>$6 mJy              & 6 mJy                 & $>$6 mJy  \\ 
  $i_{\rm AB}-W3_{\rm AB}$            & 6.0                        & 6.8                        & 6.2                        & 7.2\\
  REW \civ                                 & 209$\pm$6          & 225$\pm$3          &153$\pm$5           &  256$\pm$5\\
  Redshift $z_{\rm in}$        &  2.591                   &  2.381                    &  2.509                  &  2.356 \\  
  \civ FWHM km s$^{-1}$   & 2863$\pm$65       & 4787$\pm$52       & 4280$\pm$112   & 3989$\pm$62 \\ 
  \oiii\ FWHM erg s$^{-1}$ & 2811                      & 4971                     & 3057                    & 2625 \\ %% From Zam16
  % \oiii FWHM erg s$^{-1}$ & ---                        & 5627                  & 3057                   & 2625 \\ %% From Alexan_the
  Spectro-polarimertry       &   $\times$            &  $\surd$                &  $\surd$           & $\times$  \\
  VLA data                          & ?                            &?                             & ?                        & ?  \\ 
  ALMA  Band 6                  & tbc                        & $\surd$                & tbc                     & $\surd$  \\
  {\it HST} Cycle 24           & {\footnotesize ACS/WFC3} &{\footnotesize ACS/WFC3}    & {\footnotesize ACS/WFC3}    & {\footnotesize ACS/WFC3} \\
                                       & {\footnotesize {\bf obtained}}  & {\footnotesize {\it pending}}   & {\footnotesize {\it pending}}  & {\footnotesize {\it pending}} \\
 &&&& \\
JWST target visibility (Start) & 2019-04-01    & 2019-05-08    & 2019-05-22   & 2019-06-07  \\ 
JWST target visibility (End)  & 2019-05-07    & 2019-07-01     & 2019-07-15   & 2019-07-29   \\ 
 &&&& \\
\hline\hline
      \end{tabular}
\caption{}
\label{tab:targets} 
  \end{center}
\end{table}




%%\Huge \huge \LARGE \Large \large \normalsize (default) \small \footnotesize \scriptsize \tiny
