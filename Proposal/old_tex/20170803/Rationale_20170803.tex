
\begin{quote}
{\it ``Observations with HST have confirmed that most nearby galaxies harbor supermassive black holes in their nuclei.
How do these supermassive black holes form and evolve? Do they grow from stellar “seeds” or do they originate at the very beginning of the formation of a galaxy? These key questions are ripe for a frontal attack now. Addressing them will require the observation of active galactic nuclei (AGNs) when they first turn on, over the entire electromagnetic spectrum. With its enormous sensitivity in the infrared, NGST will be able to detect AGNs out to redshifts beyond 10.''} \\
-- Astronomy and Astrophysics in the New Millennium (2000-2010 Decadal Survey). 
\end{quote}

\noindent
Over 50 years after their formal identification, and over two decades since the calculation of their space density evolution, several fundamental facts remain unknown for high-luminosity AGN, i.e. quasars: \\
\noindent
What is the main AGN triggering mechanism at the height of quasar activity at redshifts $z=2-3$? \\ 
\noindent
What is the direct, observational evidence in individual objects, that links AGN activity to star formation? \\
\noindent
Can we observe ``AGN feedback'' in action, in situ for the most luminous sources at their peak activity?\\ 

\noindent
These remain the outstanding observational extragalactic questions of
our time. And they will be answered with the launch of the {\it James
Webb Space Telescope}.

\smallskip \smallskip
\noindent
{\it We propose the perfect Director’s Discretionary Early Release Science (DD ERS) program.
We have identified a population of infrared (20-30$\mu$m observed) bright quasars at the peak 
of cosmological quasar activity, $z\approx2.5$. 
Their global star-formation properites are currently unknown, but observations with JWST MIRI, 
in particular MRS spectroscopy, will quantify the level of star-formation in these objects. 
Moreover, the IFU aspect of the Medium Resolution Spectrometer will allow, for the first time, 
detailed investigations of the both the central AGN IR emission and potentially extended emission.
}

\smallskip \smallskip
\noindent
The recently identified Extremely Red Quasar (ERQ) objects are a
unique obscured quasar population with extreme physical conditions
related to powerful outflows across the line-forming regions. These
objects are found at the same cosmological epoch as the peak of quasar
activity, $z\approx2.5$ and are the best candidates to date to show
outflows on galactic scales; we are seeing quasar-level AGN feedback
in action, in situ. 

With our ERS MIRI observations of these ERQ, so called since they are
bright in the WISE W4 23$\mu$m band, we will deliver all the tools
necessary to the community in order to optimize Cycle 2 proposals of
5-30$\mu$m milliJansky bright sources. This is {\it a fundamental
tool} for the exploitation of a key MIRI instrument mode.

\noindent
We will satisfy the Goals and Principles of the DD ERS program by:
\begin{itemize}
\item ensuring open access to our datasets in support of the preparation of Cycle 2 proposals, and
\item engaging a broad cross-section of the astronomical community, in particular the extragalactic community, in familiarizing themselves with JWST data and scientific capabilities.
\end{itemize}

\noindent
In particular, we aim to produce several key science products:
\begin{itemize}
\item {\tt mrs\_analyzer} A Python module for analyzing MRS data; 
\item {\tt mrsfringe} A Python module for mitigating MRS fringing issues; 
\item a suite of science results on quasar feedback and evolution. 
\end{itemize}

\noindent
{\it Critically, we have already begun working closely with the MIRI team (due to the P.I.'s location at Edinburgh) and will continue to develop tools here for the MIRI Imager and MRS.}\\


\noindent
The DD ERS program is guided by the following key principles:
\begin{itemize}
\item Projects must be substantive science demonstration programs that utilize key instrument modes to provide representative scientific datasets of broad interest to researchers in major astrophysical sub-disciplines. Note that a meritorious DD ERS project need not cover every mode of the observatory. The request should match the focused science goals of the proposal.

\item Projects must design, create, and deliver science-enabling products to help the community understand JWST's capabilities.  An initial set of products must be delivered by the release of the Cycle 2 GO Call for Proposals (September 2019).  Each project must define a core team to be responsible for the timely delivery of such products according to a proposed project management plan, with performance subject to periodic review.

\item All observations must be schedulable within the first 5 months of Cycle 1 (planned to be from April to August 2019), and a substantive subset of the observations must be schedulable within the first three months.   Target lists must be flexible to accommodate possible changes to the scheduled start of science observations.

\item Both raw and pipeline-processed data will enter the public domain immediately after processing and validation at STScI. These data will have no exclusive access periods (i.e., no proprietary time).
\end{itemize}


\smallskip \smallskip
\noindent
STScI recognizes and supports the benefits of having diverse and inclusive scientific teams involved in the formulation of ERS proposals.  Programs with diverse representation of community members in a given sub-discipline helps ensure that the investigations will be of broad interest. Broad involvement also facilitates the dissemination of JWST expertise through a more extensive network, and promotes more equitable participation in JWST scientific discovery.

\smallskip \smallskip
\noindent
The DD ERS program will be essential for informing the scientific and technical preparation of Cycle 2 General Observer (GO) proposals, submitted seven months after the end of commissioning.



\smallskip
\smallskip
\noindent
{\bf \underline{The Medium-Resolution Spectrometer}:}
The JWST MIRI medium-resolution spectrometer (MRS; Wells et al. 2015) will observe simultaneous spatial and spectral information between 4.9 and 28.8 $\mu$m over a contiguous field of view up to 7.2" × 7.9" in size. This is the only JWST configuration offering medium-resolution spectroscopy (with R from 1500 to 3500) longward of 5.2 $\mu$m.  \\

\noindent
MRS observations are carried out using a set of 4 integral field units (IFUs), each of which covers a different portion of the MIRI wavelength range. MRS IFUs split the field of view into spatial slices, each of which produces a separate dispersed "long-slit" spectrum. Post-processing produces a composite 3-dimensional (2 spatial and one spectral dimension) data cube combining the information from each of these spatial slices. This process is illustrated schematically in Figure 1.
MRS operations have been designed to allow for efficient observations of point sources, compact sources, and fully extended sources. The observer will have control over 3 primary variables: (1) wavelength coverage, (2) dithering pattern, and (3) detector read out mode and exposure time (via the number of frames and integrations).

\medskip
\medskip
\smallskip
\smallskip
\noindent
{\bf \underline{Relation to Spitzer IRS}:}\\
Major Achievement of Spitzer was IRS. \\
However IRS had fringing. \\
Also, Spitzer IRS died before WISE; therefore now WISE W4 objects were observed by the IRS.\\


