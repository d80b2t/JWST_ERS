\section*{General Outline and Motvations Driving Data Processing \& Analysis}
We are already ensuring open access to representative data sets, processeing pipelines and analysis tools in 
support of the preparation of both Cycle 1 and Cycle 2 proposals. 
%engage a broad cross-section of the astronomical community in familiarizing themselves with JWST data and scientific capabilities.
The key links are:: \\ 

%\vspace{6pt}
%\noindent
%\vspace{4pt}
\href{https://github.com/d80b2t/JWST\_ERS}{{\tt github.com/d80b2t/JWST\_ERS}} 

(the P.I.'s personal and public GitHub repository).
\\

%\vspace{6pt} 
%\noindent
\href{https://github.com/miri-mrs}{{\tt github.com/miri-mrs}} 

(a ``Organizational'' public GitHub repository for the Community).
\\

%\vspace{6pt}
%\noindent
\href{http://miri-mrs.readthedocs.io/en/latest/}{{\tt miri-mrs.readthedocs.io}}

(With the source code here: \href{https://github.com/miri-mrs/ERQs/tree/master/docs}{{\tt https://github.com/miri-mrs/ERQs/tree/master/docs}}). \\


\medskip \medskip
\noindent
Our ERQ ERS proposal is the first part of a multi-cycle proposal
project and plan.  As such, we are {\it already highly motivated to
produce the data processing tools, codes, documentation and identify
the critical science-enabling products well in advance of the release
of the Cycle 2 Call for Proposals (September 2019).}

\smallskip \smallskip
\noindent
Our Science-Enabling Products team consists of five main parts::
\begin{itemize}
    \item ``Core Coders''; 
    \item Website and \href{http://miri-mrs.readthedocs.io/en/latest/}{{\tt ReadTheDocs}} Writers; 
    \item Observational Follow-up; 
    \item Senior members of staff, able to supply students/ ask for funding; 
    \item A small steering committee. 
\end{itemize}


\section*{ERS ERQ Science Enabling Products}
{\tt mrs\_analyzer} is  a Python module for analyzing MRS data; \\
{\tt mrsfringe} is a Python module for mitigating MRS fringing issues; \

\smallskip \smallskip
\noindent
And will integrate with...\\
\href{https://github.com/STScI-JWST}{\tt https://github.com/STScI-JWST} \\
see e.g. \\
{\tt https://github.com/STScI-JWST/jwst/blob/master/jwst/mrs\_imatch/mrs\_imatch\_step.py}\\


\section*{MIRISim}
MIRI data simulations (at ESA) include an Integral Field observation
with the Medium Resolution Spectrograph (MRS), a Low Resolution
Spectrograph (LRS) observation, and an imaging observation. (Credit:
ESA, Pamela Klaassen and the MIRISim Team).

\iffalse
\section{Useful links}
http://astroconda.readthedocs.io/en/latest/ \\
https://www.cosmos.esa.int/web/jwst/simulations\\
https://confluence.stsci.edu/display/JWSTDADF/JWST+Data+Analysis+Development+Forum\\
https://jwst.stsci.edu/science-planning/data-analysis-toolbox\\
https://www.youtube.com/watch?v=A024z9CITZs\\
https://jwst.stsci.edu/science-planning/proposal-planning-toolbox/simulated-data\\
\fi


\section*{Delivery Schedule for Science-Enabling Products.} 
Proposals must present a delivery schedule for science-enabling products. A description of STScI pipeline data products, processing and analysis software, and their anticipated availability, will be provided by the May 2017 release of the final version of this Call for Proposals.  Proposers may consider multiple deliveries, with more advanced products provided over longer timescales. Proposals may include the collection, processing and analysis of ancillary data as part of an integrated DD ERS proposal.


\section*{Co-Investigators and Delivery of Science-Enabling Products.}
Co-Investigators, together with the PI (and any Co-PIs) comprise a core team with the responsibility of developing and delivering science-enabling products as described in the proposal, as well as carrying out selected key aspects of the science investigations.  A Co-I must have a well-defined, and generally sustained, continuing role in team activities, serve under the direction of the PI, or co-PI(s). Co-investigators may or may not receiving funding, pending eligibility, through the DD ERS program. A bibliography and SEP tasks of our team can be found \href{https://github.com/d80b2t/JWST_ERS/blob/master/Proposal/CoI_bios.tex}{here} and the SEP Deliverables are \href{https://github.com/d80b2t/JWST_ERS/tree/master/Deliverables}{here}. 

%\subsection{Dr. Nicholas Ross}
P.I. Dr. Nic Ross is a deep believer in delivering science-enabling
products, including datasets, catalogs, analysis codes, plots,
algorithms and where possible computational resources to the wide
astronomical community.  As such, the call for delivering
science-enabling products by the release of the Cycle 2 Call for
Proposals (September 2019) is fully inline with his scientific
practice.

\smallskip \smallskip
\noindent
Ross has being developing and buidling up his GitHub Repositories over
the last year or so, \href{https://github.com/d80b2t}{\tt
github.com/d80b2t} and indeed now does all his analysis and paper
writing on GitHub.

\smallskip \smallskip
\noindent
Ross will devote a considerable amount of his personal research time
(and due to his STFC ERF has 100\% FTE for research) to leading the
developement and timely prodcution of the ERS ERQ science-enabling
products.


\subsection{Dr. David Rosario} 
Co-PI Dr. David Rosario is awesome and also loves to write code. ;-)


\subsection{Prof. David Alexander} 
Prof. Alexander is an expert in high-$z$ obsured AGN.  He will use his
considerable {\it Spitzer IRS} expereince to help test our MIRI MRS
data-analysis toolkit.


\subsection{Dr. Rachael Alexandroff} 
Dr. Alexandroff is an leading expert on the ERQ population.  She will
bring to bear her now considerable and recent data analysis (long-slit
optical, polarimerty, radio) data analysis experience to build our
MIRI MRS data-analysis toolkit.


\subsection{Dr. Richard Bielby}

\iffalse
\subsection{Prof. Beth Biller}
Prof. Biller is an expert in infrared coronagraphic observations. 
While we do not intend to use the MIRI coronagraphs in this proposal, 
longer term observations would potentially involve observing the ERQs
with the Lyot or 4QPM if this became appropriate and technically feasible. 
\fi


\subsection{Prof. Niel Brandt}


\subsection{Dr. Rob Crain}
Dr. Rob Crain is a Royal Society University Research Fellow and will 
lead the theoretical team. 


\subsection{Prof. Xiaohui Fan}
Prof. Fan is a leader in surveys of high-redshift quasars and
reionization. He has extensive experience in studying quasars and
their host galaxies with {\it HST} and {\it Spitzer}.


\subsection{Prof. Fred Hamann}


\subsection{Prof. Dale Kocevski}
Prof. Kocevski is...
We are also asking for the appriopriate level of 
post-doctoroal support for Prof. Kocevski. 

\subsection{Prof. Linhua Jiang}


\subsection{Dr. Stephanie LaMassa}
Dr. LaMassa is currently at the STScI and is already involved with the
documentation efforts there. As such, Dr. LaMassa will help with those
efforts, along with writing code and potentially leading follow-up
where approriate. She will also be a natural link to the direct
efforts of the Space Telsecop Science Institute.


\subsection{Dr. Chelsea MacLeod}


\subsection{Dr. Ian McGreer}


\subsection{Prof. Brice Menard}	


\subsection{Dr. James Mullaney}


\subsection{Prof. Adam Myers}
Prof. Myers is an expert on the statistical analysis of reddened,
obscured and optically luminous quasars. He has co-authored many
well-cited publications on targeting quasars, quasar clustering,
high-redshift and unusual quasars, and quasars in the time
domain. Prof. Myers has made follow-up observations of quasars, and
other objects, at telescopes on five continents. His work has been
funded multiple times by the NSF and NASA, including via space
telescope programs such as those for {\it Chandra} and {\it Spitzer}. He has
served on time allocation committees for GALEX and the {\it HST}. 

\smallskip \smallskip
\noindent
Prof. Myers has also worked extensively in large survey
collaborations, often in formal management roles. He is an Architect
of SDSS-III and SDSS-IV, was the quasar target selection lead for the
SDSS-IV/eBOSS survey, is the Level 3 Target Selection Manager for the
Dark Energy Spectroscopic Instrument (DESI) and is the documentation
and website lead for the Legacy Surveys
(http://legacysurvey.org). {\it Prof. Myers is a strong advocate for
transparent and reproducible science. For example, as part of his work
on DESI, he has contributed over 10,000 lines of code to publicly
visible github repositories.}


\subsection{Dr. Jessie Runnoe}
Dr. Runnoe is an expert on quasar central engines at radio through
X-ray wavelengths.  Drawing on her vast observational experience, she
will contribute to the development of the MIRI MRS data-analysis
toolkit and assist with follow-up observations of the ERQ core sample.
She will be part of the Core Coding and Observational Follow-up
groups.

\subsection{Prof. Don Schneider}
Prof. Donald Schneider has been involved with the Sloan Digital Sky
Survey since its earliest design stages in the 1980s and has
considerable experience in preparing large datasets for community use,
via leading several editions of the SDSS Quasar Catalogs and
participating in the annual public Data Releases. Prof. Schneider will
be on the follow-up Observational team, obtaining time on the HET if
necessary.


\subsection{Prof. Tom Shanks}	


\subsection{Dr. John Stott}


\subsection{Prof. Michael  Strauss}


\subsection{Dr. Renske Smit}		


\subsection{Prof. Martin Ward}		


\subsection{Prof. Gillian Wright}


\subsection{Prof. Nadia Zakamsaka} 



