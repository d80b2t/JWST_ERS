\vspace{-4pt}
\section*{{\sc Executive Summary}}
\vspace{-6pt}
%%\smallskip \smallskip
\noindent
{\it 1. Justification for ERS time}\\
%% How will your program help the community learn to do science with JWST and prepare for Cy2?  What science-enabling products are proposed for development and release? How will your program demonstrate baseline JW science capabilities?
Our Director’s Discretionary time for an Early Release Science
(DD-ERS) program will help the community learn about the longest
wavelength spectrograph on {\it James Webb Space Telescope (JWST)},
the Mid-Infrared Instrument (MIRI) Medium-resolution spectrometer
(MRS).  We will release a suite of science-enabling products (SEPs)
via a public data analysis and code repository that we have already
begun to build: \href{https://github.com/miri-mrs}{\tt
github.com/miri-mrs}.  The accompanying documentation is also already
being written: \href{http://miri-mrs.readthedocs.io/}{{\tt
miri-mrs.readthedocs.io}}.
%The third stage of the pipeline for MRS spectroscopy is called CALIFU3 
{\it Our primary SEP goal is to produce a Python package that quickly
manipulates and analyzes the full MRS Level 3 data, in particular the
MRS Spectral Cubes and 1D spectra. }
%We note there is already Python legacy code for this type of 
%analysis: {\href https://spectral-cube.readthedocs.io/}{\tt spectral-cube.readthedocs.io}.

\smallskip %\smallskip
\smallskip \smallskip
\noindent
{\it 2. Project Management Plan \& Budget} \\
%% Data processing and analysis plan, roles, responsibilities, work schedule, budget. Identification of core team responsible for science-enabling product development and delivery.
Our team members have a long and proven history of delivering SEPs, 
catalogs, data products, web access pages, documentation and the
necessary helpdesk support for large collaborations on strict
deadlines.  Our project is led by a STFC Ernest Rutherford Senior
Research Fellow, who is able to contribute 100\% FTE to the
development, delivery and management of the SEPs.  We also ask for
support for two postdoctoral researchers who will be in place at
Launch time.

\smallskip %\smallskip
\smallskip \smallskip
\noindent
{\it 3. Scientific Justification}\\
%Why are the observations scientifically compelling and require JWST?.
Our science case is simple, yet strikes at the heart of a major and
still open extragalactic astrophysical question: {\it What are the
star-formation properties of mid-infrared luminous quasars at the peak
of quasar activity? } Furthermore, we look to use the IFU capability
of MIRI MRS in order to quantify the spatial location of the IR
luminosity.  This is an ideal investigation for {\it James Webb}; no
other telescope or observatory, ground or space-based, is currently
being built or designed, to perform spectroscopy beyond 19$\mu$m;
these red wavelengths vital for accessing polycyclic aromatic
hydrocarbon (PAH) spectral features at $z>2$.

\smallskip %\smallskip
\smallskip \smallskip
\noindent
{\it 4. Description of the Observations}\\
%Establish feasibility for early execution, and flexibility in target selection to accommodate any change to start date for Cy1 science obs.
We have four primary targets; all are available for early observation.
We also have a back-up list of 10 (ten) objects, any of which would be
suitable as secondary targets and would allow us to achieve our SEP
and Science goals.


\smallskip %\smallskip
\smallskip \smallskip
\noindent
{\it 5. Team Diversity}\\
%% 	Description of how the proposing team represents and has input  	from diversity of experts with broad demographics within sub-discipline
Our team is an ensemble of observational extragalactic experts with a
broad geographical dispersion. This is a new collaboration, but with
substantial heritage and expertise from 
the SDSS, 
the {\it HST/Chandra} Deep Field surveys, 
and more recent ground-based IR IFU collaborations (e.g., VLT/KMOS).  
Our team has a gender binary split of 7:12 F:M, and is
predominantly White European or White American.
%Alexander    Alexandroff
%Brandt          Banerji
%Fan               MacLeod
%Hamann       Runnoe
%Kocevski      Smit%
%Mullaney      Zakamska
%Myers           LaMassa
%Rosario
%Ross
%Schneider
% J Stott
%M Strauss
%Crain
%McGreer
%Proin non tempus velit. Etiam laoreet, enim nec scelerisque dictum, tortor massa tempor enim, id pretium justo quam ac lectus. Maecenas diam nibh, interdum at lobortis sit amet, dignissim et quam. Sed tincidunt faucibus risus, congue tempus nisl consectetur eget. 



\section*{Science Rationale}
\vspace{-6pt}
\noindent
Over 50 years after their formal identification, and over two decades
since the calculation of their space density evolution, several
fundamental facts remain unknown for high-luminosity AGN,
i.e. quasars: What is the main AGN triggering mechanism at the height
of quasar activity at redshifts $z=2-3$? What direct,
observational evidence in individual objects links AGN activity
to star formation?  Can we observe ``AGN feedback'' in action, in situ,  
for the most luminous sources at their peak activity? 
Such unknowns about the co-evolution of black holes and their host
galaxies remain among the most fundamental unanswered questions in
extragalactic astronomy.
%These remain the outstanding observational extragalactic questions of our time. 
And they will be answered with the launch of the {\it James Webb Space
Telescope}.

\smallskip \smallskip
\noindent
We have identified a population of obscured, mid-infrared bright
quasars at the peak of cosmological quasar activity.  These sources are
mid-IR luminous and may be powered by major bursts of star formation
tied to an early phase of galaxy evolution/formation. However, their
global star-formation properties are currently unknown, but
observations with JWST MIRI, and in particular MRS spectroscopy, will
quantify the level of star-formation in these objects.  {\it We will
observe four ``extremely red'' quasars with MIRI MRS across the full wavelength range to
very high signal to noise.} These observations will address the
fundamental question of the link between star-formation and AGN
activity, at the cosmological epoch for both of these processess; an
investigation JWST was specifically built for.

\smallskip \smallskip
\noindent
The recently identified Extremely Red Quasar (ERQ) objects are a
unique obscured quasar population with extreme physical conditions
related to powerful outflows across the line-forming regions. These
objects are found at the same cosmological epoch as the peak of quasar
activity, $z\approx2.5$, and are the best candidates to date to show
outflows on galactic scales in high-luminosity objects at these epochs. 

\smallskip \smallskip
\noindent
However, it is unknown whether the large IR luminosities observed in
these quasars is from star formation, which would produce strong
polycyclic aromatic hydrocarbon (PAH) spectral features, or if the hot
dust near the central quasar, which should produce much weaker/no PAH
emission (due to the AGN MIR emission diluting and even destroying PAH
features).  is driving the MIR luminosity.
{\it Via the detection, or otherwise, of PAH spectral features, we will 
measure the SFRs during what is potentially a very early/obscured stage 
of massive galaxy formation in the Extremely Red Quasar population.}

\smallskip \smallskip
\noindent
MIRI MRS is the instrument of choice since no other medium resolution
spectrometer on JWST observes longward of 5$\mu$m; going redder than
this is crucial in order to detect PAH features in $z>2$ objects.  
If present we will observe the most prominent, well-known major PAH emission 
features at 3.3, 6.2, 7.7, and 8.6$\mu$m. The mid-IR spectral region also 
presents a suite of high-ionization lines: 
%prominent forbidden ionic emission lines, 
\snine\ at 1.252$\mu$m, \six\ at 1.430$\mu$m, \sixi\ at 1.932$\mu$m,  
\sivi\ at 1.962$\mu$m, \caviii\ at 2.321$\mu$m, \sivi\ at 2.483$\mu$m 
\siix\ at 3.935$\mu$m and \arii\ at 6.97$\mu$m. 
% such as [Ne ii]12.8 μm, [Ne v]14.3 μm, [Ne iii]15.5 μm, [S iii]18.7 μm and 33.48 μm, [O iv]25.89 μm and [Si ii]34.8 μm (e.g
{\it The 
desire to immediately gain high signal-to-noise spectra in order to
investigate the physics and chemistry of quasar PAHs, along with
observational overhead concerns, pushes us to observe each object for
3.6hours, for a total program Charged Time of 22.20 hours.}



\section*{Community Access Rationale}
\vspace{-6pt}
\noindent
We will satisfy the Goals and Principles of the DD ERS program of
ensuring open access to our datasets and analyses in support of the
preparation of Cycle 2 proposals, and, engaging a broad cross-section
of the astronomical community, in particular the extragalactic
community, in familiarizing themselves with JWST data and scientific
capabilities by utilizing modern analysis (code writing) repository
tools and documentation that is standardized and publicly editable.

\smallskip \smallskip
\noindent
Moreover, we have already begun to design and create science-enabling
products (SEPs) to help the community understand JWST's capabilities,
Our \href{https://github.com/miri-mrs}{{\tt MIRI MRS Code Repo}} is
already active and completely accessible to anyone in the broader
community.  We will deliver the MIRI MRS SEPs first with mock data
before the launch of JWST, and then in rapid fashion once the start of
science operations commences in April 2019.
%%
Our team's commitment to an open access ideology, not only for data,
but for analysis codes, documentation, and scientific manuscripts is
already evident and in place (for an example, see
\href{https://github.com/d80b2t}{{\tt the P.I.'s GitHub}}).  Via these
code repos, and extensive documentation efforts, we aim to engage a
broad cross-section of the astronomical community, in particular the
extra-galactic community, in familiarizing themselves with JWST data
and its scientific capabilities.

%\smallskip \smallskip
%\noindent
%In particular, we aim to produce several key science products:
%\begin{itemize}
%\item {\tt mrs\_analyzer} A Python module for analyzing MRS data; 
%\item {\tt mrsfringe} A Python module for mitigating MRS fringing issues; 
%\item a suite of science results on quasar feedback and evolution. 
%\end{itemize}

\noindent
{\it Critically, we have already begun working closely with the MIRI
team (due to the P.I.'s location at Edinburgh) and will continue to
develop tools here for the MIRI MRS.}
%%
We will produce of SEP analysis code and documentation.  Critically,
with {\it a maximum} of 6 months between the first ERS observations
(April 2019) and the Cycle 2 GO Call for Proposals (September 2019),
this will be too short for dissemination of our findings, novel
techniques and science results in the traditional manner, i.e. via
published journal articles. Moreover, ongoing updated versions of our
analyses are envisaged to happen until right up to the Cycle 2
deadline.  To solve these issues, we will fully employ the power of a
code version repository system, in our case GitHub, to keep the
community informed and updated with or SEPs. GitHu {\it has code
versioning automatically built-in} so proper referencing of
e.g. technical notes is straight-forward.






