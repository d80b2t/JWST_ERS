We propose the perfect Director’s Discretionary Early Release Science
(DD ERS) program. We have identified a population of mid-infrared
bright quasars at the peak of cosmological quasar activity. Their
global star-formation properites are currently unknown, but
observations with JWST MIRI, and in particular MRS spectroscopy, will
quantify the level of star-formation in these objects.
%%
Moreover, we have already begun to design and create science-enabling
products (SEPs) to help the community understand JWST's capabilities,
Our \href{https://github.com/miri-mrs}{{\tt MIRI MIRS Code Repo}} is
already active and completely accessible to anyone in the broader
community.  We will deliver the MIRI MRS SEPs first with mock data
before the launch of JWST, and then in rapid fashion once the start of
science operations commences in April 2019.

\smallskip \smallskip
\noindent
Our teams commitment to an open access idealogy, not only for data,
but for analysis codes, documentation, and scientific manuscripts is
already evident and in place (see also
\href{https://github.com/d80b2t}{{\tt the P.I.s GitHub}}).  We aim to
engage a broad cross-section of the astronomical community, in
particular the extra-galactic community, in familiarizing themselves
with JWST data and scientific capabilities.

\smallskip \smallskip
\noindent
In this proposal we first outline the Scientifc and Community Access rational. 
We give details of the scientific motivation in the Scientific Justification and why our quasars are the ideal ERS targets. 
Details of our observations are given in the Technical Description. In short we will observe four quasars with MIRI MRS across the full wavelenght range. The exposure time for each object is $xxxx$ seconds, and our entire program (with APT Smart Accounting envoked) is 25.00 hours. 
We do give a list of Alternative Targets; We do not have and special observatioal requirements, nor any
Coordinated Parallel Observations, and no duplications.
We give more details of our plan to deliver the SEPs in the Data Processing \& Analysis Plan section. This includes a list of analysis code modules we will develop, how the community can rapidly access our findings and tech notes and the breakdown of who in the Core Anaylsis Team is going to do what. 


\section*{Science Rationale}
%\subsection{Science Rationale}
\iffalse
\begin{quote}
{\it ``Observations with HST have confirmed that most nearby galaxies harbor supermassive black holes in their nuclei.
How do these supermassive black holes form and evolve? Do they grow from stellar “seeds” or do they originate at the very beginning of the formation of a galaxy? These key questions are ripe for a frontal attack now. Addressing them will require the observation of active galactic nuclei (AGNs) when they first turn on, over the entire electromagnetic spectrum. With its enormous sensitivity in the infrared, NGST will be able to detect AGNs out to redshifts beyond 10.''} \\
-- Astronomy and Astrophysics in the New Millennium (2000-2010 Decadal Survey). 
\end{quote}
\fi

\smallskip \smallskip
\noindent
Over 50 years after their formal identification, and over two decades since the calculation of their space density evolution, several fundamental facts remain unknown for high-luminosity AGN, i.e. quasars: What is the main AGN triggering mechanism at the height of quasar activity at redshifts $z=2-3$? What is the direct, observational evidence in individual objects, that links AGN activity to star formation?  Can we observe ``AGN feedback'' in action, in situ for the most luminous sources at their peak activity? These remain the outstanding observational extragalactic questions of our time. And they will be answered with the launch of the {\it James Webb Space Telescope}.

\smallskip \smallskip
\noindent
The recently identified Extremely Red Quasar (ERQ) objects are a
unique obscured quasar population with extreme physical conditions
related to powerful outflows across the line-forming regions. These
objects are found at the same cosmological epoch as the peak of quasar
activity, $z\approx2.5$ and are the best candidates to date to show
outflows on galactic scales; we are seeing quasar-level AGN feedback
in action, in situ. 

\smallskip \smallskip
\noindent
With our ERS MIRI observations of these ERQ, so called since they are
bright in the WISE W4 23$\mu$m band, we will deliver all the tools
necessary to the community in order to optimize Cycle 2 proposals of
5-30$\mu$m milliJansky bright sources. This is {\it a fundamental
tool} for the exploitation of a key MIRI instrument mode.

\smallskip
\smallskip
\noindent
{\bf \underline{The Medium-Resolution Spectrometer}:}
The JWST MIRI medium-resolution spectrometer (MRS; Wells et al. 2015) will observe simultaneous spatial and spectral information between 4.9 and 28.8 $\mu$m over a contiguous field of view up to 7.2" × 7.9" in size. This is the only JWST configuration offering medium-resolution spectroscopy (with R from 1500 to 3500) longward of 5.2 $\mu$m.  \\
%the IFU aspect of the Medium Resolution Spectrometer will allow, for the first time, detailed investigations of the both the central AGN IR emission and potentially extended emission.

\noindent
MRS observations are carried out using a set of 4 integral field units
(IFUs), each of which covers a different portion of the MIRI
wavelength range. MRS IFUs split the field of view into spatial
slices, each of which produces a separate dispersed "long-slit"
spectrum. Post-processing produces a composite 3-dimensional (2
spatial and one spectral dimension) data cube combining the
information from each of these spatial slices. This process is
illustrated schematically in Figure 1.  MRS operations have been
designed to allow for efficient observations of point sources, compact
sources, and fully extended sources. The observer will have control
over 3 primary variables: (1) wavelength coverage, (2) dithering
pattern, and (3) detector read out mode and exposure time (via the
number of frames and integrations).



\section*{Community Access Rationale}
\noindent
We will satisfy the Goals and Principles of the DD ERS program by:
\begin{itemize}
\item ensuring open access to our datasets in support of the preparation of Cycle 2 proposals, and
\item engaging a broad cross-section of the astronomical community, in particular the extragalactic community, in familiarizing themselves with JWST data and scientific capabilities.
\end{itemize}

\noindent
In particular, we aim to produce several key science products:
\begin{itemize}
\item {\tt mrs\_analyzer} A Python module for analyzing MRS data; 
\item {\tt mrsfringe} A Python module for mitigating MRS fringing issues; 
\item a suite of science results on quasar feedback and evolution. 
\end{itemize}

\noindent
{\it Critically, we have already begun working closely with the MIRI team (due to the P.I.'s location at Edinburgh) and will continue to develop tools here for the MIRI Imager and MRS.}\\

\noindent
We will produce of SEP analyse code and documentation.  However, the 7
month period between the end of commissioning and Cycle 2 proposal
deadlines will be too short for disemination of our findings, novel
techniques and potentially even science results in the traditional
manner (via journals). Moreover,  continuall update versions 
of our analyses are envisaged to happen until right up to the Cycle 2 deadline. 

To solve these issues, we will fully emply the power of a code version
repository system, in our case GitHub, to keep the community informed
and updated with or SEPs. GitHu {\it has code versioning automatically
built-in} so proper referencing of e.g. technical notes can easily


\noindent
The DD ERS program is guided by the following key principles:
\begin{itemize}
\item Projects must be substantive science demonstration programs that utilize key instrument modes to provide representative scientific datasets of broad interest to researchers in major astrophysical sub-disciplines. Note that a meritorious DD ERS project need not cover every mode of the observatory. The request should match the focused science goals of the proposal.

\item Projects must design, create, and deliver science-enabling products to help the community understand JWST's capabilities.  An initial set of products must be delivered by the release of the Cycle 2 GO Call for Proposals (September 2019).  Each project must define a core team to be responsible for the timely delivery of such products according to a proposed project management plan, with performance subject to periodic review.

\item All observations must be schedulable within the first 5 months of Cycle 1 (planned to be from April to August 2019), and a substantive subset of the observations must be schedulable within the first three months.   Target lists must be flexible to accommodate possible changes to the scheduled start of science observations.

\item Both raw and pipeline-processed data will enter the public domain immediately after processing and validation at STScI. These data will have no exclusive access periods (i.e., no proprietary time).
\end{itemize}




