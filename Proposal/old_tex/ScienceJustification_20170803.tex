\newcommand{\imw}{$i$--$W3$}
\newcommand{\imwf}{$i$--$W4$}
\newcommand{\rmwf}{$r$--$W4$}
\newcommand{\imwt}{$i$--$W2$}
\newcommand{\wtmwf}{$W3$--$W4$}
%\newcommand{\kms}{km s$^{-1}$}
\newcommand{\cmN}{cm$^{-2}$}
\newcommand{\cmn}{cm$^{-3}$}
\newcommand{\msun}{M$_{\odot}$}
\newcommand{\lsun}{L$_{\odot}$}
\newcommand{\lam}{$\lambda$}
\newcommand{\mum}{$\mu$m}
\newcommand{\ebv}{$E(B$$-$$V)$}
\newcommand{\heii}{\mbox{He\,{\sc ii}}}
\newcommand{\cv}{\mbox{C\,{\sc v}}}
\newcommand{\civ}{\mbox{C\,{\sc iv}}}
\newcommand{\ciii}{\mbox{C\,{\sc iii}}}
\newcommand{\cii}{\mbox{C\,{\sc ii}}}
\newcommand{\nv}{\mbox{N\,{\sc v}}}
\newcommand{\niv}{\mbox{N\,{\sc iv}}}
\newcommand{\niii}{\mbox{N\,{\sc iii}}}
\newcommand{\ovii}{\mbox{O\,{\sc vii}}}
\newcommand{\ovi}{\mbox{O\,{\sc vi}}}
\newcommand{\ov}{\mbox{O\,{\sc v}}}
\newcommand{\oiv}{\mbox{O\,{\sc iv}}}
\newcommand{\oiii}{\mbox{O\,{\sc iii}}}
\newcommand{\oii}{\mbox{O\,{\sc ii}}}
\newcommand{\oi}{\mbox{O\,{\sc i}}}
\newcommand{\feii}{\mbox{Fe\,{\sc ii}}}
\newcommand{\feiii}{\mbox{Fe\,{\sc iii}}}
\newcommand{\mgii}{\mbox{Mg\,{\sc ii}}}
\newcommand{\neviii}{\mbox{Ne\,{\sc viii}}}
\newcommand{\aliii}{\mbox{Al\,{\sc iii}}}
\newcommand{\siiv}{\mbox{Si\,{\sc iv}}}
\newcommand{\siiii}{\mbox{Si\,{\sc iii}}}
\newcommand{\siii}{\mbox{Si\,{\sc ii}}}
%\newcommand{\lya}{\mbox{Ly$\alpha$}}
%\newcommand{\lyb}{\mbox{Ly$\beta$}}
\newcommand{\hi}{\mbox{H\,{\sc i}}}


One lasting scientific legacy of the {\it Hubble Space Telescope
(HST)} will be the discovery of giant black holes at the centers of
galaxies, confirming the longstanding theory of the ``central
engines'' of quasars. One of the major surprises from the {\it Hubble}
was the discovery of a correlation between black hole mass and galaxy
properties.  This connection, causal or otherwise, may provide crucial
clues to how and why these black holes formed and how their host
galaxies evolved. {\it As of the launch of the James Webb Space
Telescope (JWST), this is one of the outstanding questions in
astrophysics.}
%%Assessment of Options for Extending the Life of the Hubble Space Telescope: Final Report (2005)
%% https://www.nap.edu/read/11169/chapter/5

Furthermore, discovery of spectral lines in active galaxies reveals that black holes can trigger massive star formation. 
As such, %the two main energy sources available to a galaxy are nuclear fusion in stars and gravitational accretion onto compact objects. 
%%
The link between massive galaxies and the central super-massive black holes (SMBHs) that seem ubiquitous in them is now thought to be vital to the understanding of galaxy formation and evolution ([1], [2]), and huge observational and theoretical effort has been invested in trying to measure and understand the physics involved in these enigmatic systems.
The two main energy sources available to a galaxy are nuclear fusion in stars and gravitational accretion onto compact objects, and we still do not fully understnad the interaction of an active galactic nuclei and the star formation properties of their host galaxies, {\it especially at high-z, at the height of the SFR and quasar activity.} 


\smallskip
\smallskip
\noindent
Hubble (more or less) discovered $M-\sigma$. \\
Gives rise to the idea of AGN/QSO feedback (in order to shape the LF at the high-mass end)\\
However, {\it direct observational evidence} for AGN feedback is conspicuous by its absence. This is especially true at high-$z$, e.g. $z=2-3$, at the height of the Quasar Epoch. \\
We have identified the best candidates that suggest we are seeing quasar feedback in action, in situ at high-redshift. These are the ``Extremely Red Quasars'' identified via their WISE W3/4 colors. \\
As such, these milliJansky luminous AGN {\it are ideal targets for JWST MIRI}. 


\medskip
\medskip

\smallskip
\smallskip
\noindent
{\bf \underline{The Extremely Red Quasar Population}:}
By matching the quasar catalogues of the Sloan Digital Sky Survey
(SDSS), the Baryon Oscillation Spectroscopic Survey (BOSS) to the
Wide-Field Infrared Survey Explorer (WISE), Ross et al. (2015)
discovered quasars with extremely red infrared-to-optical colours:
$r_{\rm AB}-W4_{\rm Vega}>14$ mag, i.e., $F_\nu({\rm 22\mu
m})/F_\nu(r) \gtrsim 1000$.  These objects have infrared luminosities
$\sim 10^{47}$ erg s$^{-1}$.  The original motivation here was to look
for PAHs that had redshifted in the WISE W4 band for $z\approx$2.5
quasars in order to study and link luminous AGN activity and star
formation at the ``height of the quasar epoch''.

Hamann et al. (2017) then fully and properly refined the selection of
the ERQs, changed the definition based on other data and common
properties, and indeed found many more objects in this new scheme. The
ERQs have a suite of peculiar emission-line properties including large
rest equivalent widths (REWs), unusual ``wingless'' line profiles,
large \nv /\lya , \nv /\civ , \siiv /\civ\ and other flux ratios, and
very broad and blueshifted [\oiii ] \lam 5007.

In Hamann et al., our team identified a ``core'' sample of 97 ERQs
with nearly uniform peculiar properties selected via \imw\ $\ge 4.6$
(AB) and REW(\civ ) $\ge$ 100 \AA\ at redshifts 2.0--3.4. A broader
search finds 235 more red quasars with similar unusual
characteristics. The core ERQs have median luminosity $\left<\log L
({\rm ergs/s})\right> \sim 47.1$, sky density 0.010 deg$^{-2}$,
surprisingly flat/blue UV spectra given their red UV-to-mid-IR colors,
and common outflow signatures including BALs or BAL-like features and
large \civ\ emission-line blueshifts. Their SEDs and line properties
are inconsistent with normal quasars behind a dust reddening
screen. These ``Core ERQs'' are a unique obscured quasar population
with extreme physical conditions related to powerful outflows across
the line-forming regions. Patchy obscuration by small dusty clouds
could produce the observed UV extinctions without substantial UV
reddening.


\smallskip
\smallskip
\noindent
In Zakamska et al. (2016) we used XShooter/VLT to measure rest-frame
optical spectra of four $z\sim 2.5$ extremely red quasars. We
discovered very broad (full width at half max$= 2600-5000$ km
s$^{-1}$), strongly blue-shifted (by up to 1500 km s$^{-1}$)
\oiii$\lambda$5007\AA\ emission lines in these objects. In a large
sample of type 2 and red quasars, \oiii kinematics are positively
correlated with infrared luminosity, and the four objects in our
sample are on the extreme end both in \oiii kinematics and infrared
luminosity.
%%
As such, we estimate that at least 3\% of the bolometric luminosity in
these objects is being converted into the kinetic power of the
observed wind. Photo-ionization estimates suggest that the \oiii
emission might be extended on a few kpc scales, which would suggest
that the extreme outflow is affecting the entire host galaxy of the
quasar. These sources may be the signposts of the most extreme form of
quasar feedback at the peak epoch of galaxy formation, and may
represent an active ``blow-out'' phase of quasar evolution. 


\smallskip
\smallskip
\noindent
Alexandroff et al. (2017 submitted and in prep.).... \\



\medskip
\medskip
\smallskip
\smallskip
\noindent
{\bf \underline{MIR spectroscopy}:}
Fom Veilleux arXiv:0201118v1:: 
How do galaxies form? How do they evolve? How do supermassive black holes fit in this picture of galaxy formation? Which objects are the main contributors to the overall energy budget of the universe? To prop- erly answer these questions, one will need to differentiate objects powered by nuclear fusion in stars (i.e. normal and starburst galaxies) from objects pow- ered by mass accretion onto supermassive black holes (quasars and AGNs). A wide variety of diagnostic tools have been used in the past for this purpose with different degree of success.

Direct spectroscopy searches for the presence of the broad recombination lines at wavelengths where the effects of dust extinction are reduced.
We follow ``Veilleux's Commandments'':
\begin{itemize}
\item Thou shalt use lines which emphasize the differences between H II regions and AGNs; i.e., use high-ionization lines or low-ionization lines produced in the partially ionized zone. 
\item Thou shalt use strong lines which are easy to measure in typical spectra.
\item Thou shalt avoid lines which are badly blended with other emission or absorption line features.
\item Thou shalt use lines with small wavelength separation to minimize sensitivity to reddening.
\item Thou shalt use line ratios from the same elements or involving hydrogen recombination lines to eliminate or reduce abundance dependence.
\item Thou shalt avoid lines from Mg, Si, Ca, Fe – depleted onto dust grains. 
%\item Thou shalt use lines easily accessible to current UV/optical/IR detectors. 
\item Thou shalt avoid lines affected by strong stellar absorption features. 
\item Thou shalt avoid lines affected by strong atmospheric features.
\item Thou shalt use lines at long wavelengths to reduce the effects of dust extinction.
\end{itemize}


{\bf From Padovani et al.,  arXiv:1707.07134v1}
%%3.4 MIR spectroscopy
IR spectroscopy, particularly with the InfraRed Spectrograph (IRS;
Houck et al. 2004) on board the Spitzer Space Telescope, provided new
insights into the physics and classification of AGN. The unambiguous
observations of the sil icate feature at 9.7 $\mu$m in emission in many
known AGN (Hao et al., 2005; Siebenmorgen et al., 2005; Sturm et al.,
2005; Buchanan et al., 2006; Shi et al., 2006) came as the long sought
confirmation of the unified scheme. At the same time, however, IRS
observations indicated that in some cases the source of obscuration
resides in the host rather than the torus (e.g. Goulding et al., 2012;
Hatziminaoglou et al., 2015).  Identification through MIR spectroscopy
is very powerful, allowing to detect obscured AGN components even
when the MIR is dominated by the host galaxy. Several classification
diagrams have been developed to determine the AGN contribution to an
observed spectrum based on certain spectral features, such as high
ionisation emission lines like \nev, \neii and \oiv, the EW of
PAH features and the strength of the silicate feature at 9.7 $\mu$m (see,
e.g. Spoon et al., 2007; Armus et al., 2007;
Veilleux et al., 2009; Hernan-Caballero \& Hatziminaoglou, 2011). A
number of techniques have also been developed to model the observed
MIR spectra and constrain the AGN and starburst contributions (see
e.g. Schweitzer et al., 2008; Nardini et al., 2008; Deo et al., 2009;
Feltre et al., 2013).  Although MIR spectroscopy has had a great
impact on our understanding of AGN, the number of objects studied
through these techniques is limited when compared to photometric
studies, as spectroscopic observations require significantly longer
integration times. Ground-based observations are generally limited
to the brightest targets due to the effects of the Earth's atmosphere
(e.g. Alonso-Herrero et al., 2016), while deeper observations were
possible with the IRS during its cryogen-cooled phase. For the most
part, such observations were limited to z 1 luminous IR galaxies
(LIRGs), ultraluminous IR galaxies (ULIRGs), and quasars (Hernan-Caballero \& Hatziminaoglou, 2011, and references therein) although a
number of higher redshift ULIRGs were also studied by IRS (see
e.g. Kirkpatrick et al., 2012). The impact of these techniques will be
greatly expanded by the upcoming JWST (Gardner et al. 2006) 
(and Space Infrared Telescope for Cosmology and Astrophysics, SPICA; Nakagawa
et al. 2015), 
that will probe significantly fainter targets and will
allow us to select new, currently inaccessible, sets of objects, as
discussed next.


\medskip
\medskip
\smallskip
\smallskip
\noindent
{\bf \underline{Polycyclic Aromatic Hydrocarbons}:}
Polycyclic Aromatic Hydrocarbons (PAHs) are abundant, ubiquitous, and
a dominant force in the interstellar medium of galaxies (see e.g.,
Tielens, 2008, ARAA, 46, 289 for a review).  Aromatic features are
already a significant component of dusty galaxy spectra as early as
$z\approx2$ (Yan et al., 2005, ApJ, 628, 604).  and the infrared (IR)
emission features at 3.3, 6.2, 7.7, 8.6, and 11.3 $\mu$m are generally
attributed to IR fluorescence from (mainly) far-ultraviolet (FUV)
pumped large polycyclic aromatic hydrocarbon (PAH) molecules. As such,
these features trace the FUV stellar flux and are thus a measure of
star formation (Peeters et al, 2004, ApJ, 613, 986).
%%
Given the redshift of our ERQs and the MIRI wavelength coverage we will coverage $1.36 \leq \lambda_{\rm emitted} \leq 8.6 \mu$m.


\medskip
\medskip
\smallskip
\smallskip
\noindent
{\bf \underline{Relation to Spitzer IRS}:}
Major Achievement of Spitzer was IRS. \\
$R\sim600$, now $R\sim2000s$, which allows {\it chemistry.}\\
Spitzer IRS died before WISE; therefore now WISE W4 objects were observed by the IRS.\\
i.e. no $z\sim2.5$ ERQs w/ feedback in action were observed. \\



%\clearpage
\iffalse
Comparison to Spitzer IRS (as much for NPRs guide than anything!!) 
\begin{table}
\begin{center}
\begin{tabular}{ || l | c | c  || }
\hline\hline 
                                     & Spitzer IRS     & JWST MRS \\
\hline
Wavelength /$\mu$m     & 5.2 -- 38   & 5.0 -- 28.5 \\
\hline\hline 
\end{tabular}
\end{center}
\end{table}
\fi


\medskip
\medskip
\smallskip
\smallskip
\noindent
{\bf \underline{MIRI Imaging}:}
Imaging of the ERQs will tell us what environments they live in (currently totally unknown).\\
Just look at the ERQs: flux from central source $\Rightarrow$ AGN; flux from extended 
$\Rightarrow$ SF; Big puzzle since Spitzer (e.g. and cf. the submm population). \\



\medskip
\medskip
\smallskip
\smallskip
\noindent
{\bf \underline{Next Steps}:}
%I think the easiest useful thing we could get from NIR spectra is a test for any PAH emission at all. The mid-IR emission is booming in the ERQs. The PAHs would show that it's dominated by star formation. But my guess is that it’s dominated by hot dust in the torus, so no PAHs. It’s an interesting test either way, but to make it more appealing we should think about other lines available in the MIR. I don’t know anything about JWST capabilities, but if we can search for some forbidden lines used in local ULIRG/AGN studies, we could say a lot more about the kinematics and what powers the lines. 
% This paper by Veilleux seems like a good guide for lines that might be available: https://arxiv.org/abs/astro-ph/0201118


\clearpage


