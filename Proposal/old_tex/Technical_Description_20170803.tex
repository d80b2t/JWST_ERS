%%%%%%%%%%%%%%%%%%%%%%%%%%%%%%%%%%%%%%%%%%%%%%
%%
%%    D e s c r i p t i o n    o f     t h e      O b s e r v a t i o n s  : 
%%
%%%%%%%%%%%%%%%%%%%%%%%%%%%%%%%%%%%%%%%%%%%%%%

Describe the targets and observational modes to be used. Quantitative
estimates must be provided of the accuracy required to achieve key
science goals. Proposers must demonstrate that all observations can
execute in the first 5 months of Cycle 1 (planned to be from April to
August 2019), and that a substantive subset of the observations are
accessible in the first 3 months. This description should also include
the following::

\begin{enumerate}[label=\alph*]
    \item{Plan for Alternative Targets: As described in JWST DD ERS
        Special Observational Policies, proposers should qualitatively
        describe the availability of alternate targets and the process used to
        identify those targets should the start of science observations be
        delayed.  Robust ERS programs involve science investigations that can
        be performed with a variety of different targets and observations. }
      
    \item{Special Observational Requirements (if any): Justify any
        special scheduling requirements, e.g., time-critical observations.}

    \item{Justification of Coordinated Parallels (if any): Proposals
        that include coordinated parallel observations should provide a
        scientific justification for and description of the parallel
        observations. It should be clearly indicated whether the parallel
        observations are essential to the interpretation of the primary
        observations or the science program as a whole, or whether they
        address partly or completely unrelated issues. The parallel
        observations are subject to scientific review, and can be rejected
        even if the primary observations are approved.}

    \item{Justification of Duplications (if any): as detailed in the JWST
        DD ERS Proposal Policies and the JWST Duplicate Observations Policy,
        observations taken as part of the DD ERS program cannot duplicate
        those specified for the GTO Cycle 1 Reserved Observation Catalog
        (planned for release on June 15, 2017). Any duplicate observations
        must be explicitly justified.}
\end{enumerate}


\clearpage
\begin{tabular}{||  l|l|l|l|l ||}
\hline\hline
 &&&& \\
Object Name (SDSS)        & J0834+0159         &  J1232+0912          & J2215-0056        & J2323-0100 \\
 &&&& \\
\hline
 &&&& \\
Object R.A.                      & 08:34:48.48         & 12:32:41.73           & 22:15:24.00          & 23:23:26.17     \\
object declination           & $+$01:59:21.1     & $+$09:12:09.3      & $-$00:56:43.8      & $-$01:00:33.1  \\
$r$-band AB magnitude   & 21.20$\pm$0.05  & 21.11$\pm$ 0.05  & 22.27$\pm$0.12  & 21.62$\pm$ 0.08 \\  
Redshift $z_{\rm in}$        &  2.591                   &  2.381                    &  2.509                  &  2.356 \\  
CIV FWHM km s$^{-1}$   & 2863$\pm$65       & 4787$\pm$52       & 4280$\pm$112   & 3989$\pm$62 \\ 
\oiii\ FWHM erg s$^{-1}$ & 2811                      & 4971                     & 3057                    & 2625 \\ %% From Zam16
%\oiii FWHM erg s$^{-1}$ & ---                        & 5627                  & 3057                   & 2625 \\ %% From Alexan_the
Spectro-polarimertry       &   $\times$            &  $\surd$                &  $\surd$           & $\times$  \\
VLA data                          & ?                            &?                             & ?                        & ?  \\ 
ALMA  Band 6                  & tbc                        & $\surd$                & tbc                     & $\surd$  \\
{\it HST} Cycle 24           & {\footnotesize ACS/WFC3} &{\footnotesize ACS/WFC3}    & {\footnotesize ACS/WFC3}    & {\footnotesize ACS/WFC3} \\
                                       & {\footnotesize obtained}  & ?  & ?  & ? \\
 &&&& \\
JWST target visibility (Start) & 2019-04-01    & 2019-05-08    & 2019-05-22   & 2019-06-07  \\ 
JWST target visibility (End)  & 2019-05-07    & 2019-07-01     & 2019-07-15   & 2019-07-29   \\ 
 &&&& \\
\hline\hline
\end{tabular}


%%\Huge \huge \LARGE \Large \large \normalsize (default) \small \footnotesize \scriptsize \tiny
