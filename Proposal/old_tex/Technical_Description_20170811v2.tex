%%%%%%%%%%%%%%%%%%%%%%%%%%%%%%%%%%%%%%%%%%%%%%
%%
%%    D e s c r i p t i o n    o f     t h e      O b s e r v a t i o n s  : 
%%
%%%%%%%%%%%%%%%%%%%%%%%%%%%%%%%%%%%%%%%%%%%%%%

\smallskip \smallskip
\noindent
We propose to use MIRI MRS to observe four quasars across the full 
MRS wavelength range in order to deliver to the community Science 
Enabling Products (SEPs) related to the acculmation, reduction and 
timely analysis of the only spectrometer longward of 12 $\mu$m 
and the only medium-resolution spectrometer longward of 
. Details of our
four quasar targets, including the current state of our extensive
multiwavelength follow-up programs, are given in
Table~\ref{tab:targets}.

\smallskip \smallskip
\noindent
{\bf \underline{Overall Experimental Design}:} 
Our goals are two-fold for the ERS: ensure open access to representative datasets in support of the preparation of Cycle 2 proposals, and engage a broad cross-section of the astronomical community in familiarizing themselves with JWST data and scientific capabilities.
%%
As such, we have honed in on a specific early release science case that will engage a broad cross-section of the astronomical community, and deliver a dataset for MIRI MRS that can be used for e.g. extragalactic galaxy and AGN studies in Cycle 2. 
%%
After discussion with the MIRI Team (A. Glasse; priv. comm.), we settled on the ideology of picking out one
instrument (MIRI) and one observing mode (MRS) and making sure we deliver the highest quality data analysis and SEPs here for the community. Observations with MIRI MRS will also directly answer the science questions we've posed. 

\smallskip \smallskip
\noindent
Describe the overall experimental design of the program, justifying the selection of instruments, modes, and requirements. Describe how the observations contribute to the goals described in the scientific justification. Quantitative estimates must be provided of the accuracy required to achieve key science goals. Proposers must demonstrate that all observations can execute in the first 5 months of Cycle 1 (planned to be from April to August 2019), and that a substantive subset of the observations are accessible in the first 3 months. This description should also include the following,


\begin{table}
\begin{center}
\begin{tabular}{||  l|l|l|l|l ||}
  \hline\hline
  &&&& \\
  Object Name (SDSS)        & J0834+0159         &  J1232+0912          & J2215-0056        & J2323-0100 \\
  &&&& \\
  \hline
  &&&& \\
  Object R.A.                             & 08:34:48.48         & 12:32:41.73           & 22:15:24.00          & 23:23:26.17     \\
  object declination                  & $+$01:59:21.1     & $+$09:12:09.3      & $-$00:56:43.8      & $-$01:00:33.1  \\
  $r$-band AB magnitude         & 21.20$\pm$0.05  & 21.11$\pm$ 0.05  & 22.27$\pm$0.12  & 21.62$\pm$ 0.08 \\  
  WISE W4-band Vega magnitude & 6.88$\pm$0.09  & 6.78 $\pm$0.09   & 7.91$\pm$0.24  & 7.76$\pm$0.22 \\  
  WISE W4-band flux, $F_{\nu}$   & $>$6 mJy             & $>$6 mJy              & 6 mJy                 & $>$6 mJy  \\ 
  $i_{\rm AB}-W3_{\rm AB}$            & 6.0                        & 6.8                        & 6.2                        & 7.2\\

  Redshift $z$        &  2.591                   &  2.381                    &  2.509                  &  2.356 \\  
  &&&& \\
  REW \civ                                 & 209$\pm$6          & 225$\pm$3          &153$\pm$5           &  256$\pm$5\\  
\civ FWHM km s$^{-1}$   & 2863$\pm$65       & 4787$\pm$52       & 4280$\pm$112   & 3989$\pm$62 \\ 
  \oiii\ FWHM erg s$^{-1}$ & 2811                      & 4971                     & 3057                    & 2625 \\ %% From Zam16
  % \oiii FWHM erg s$^{-1}$ & ---                        & 5627                  & 3057                   & 2625 \\ %% From Alexan_the
  &&&& \\
  Spectro-polarimertry       &   $\times$            &  $\surd$                &  $\surd$           & $\times$  \\
  VLA data                          & ?                            &?                             & ?                        & ?  \\ 
  ALMA  Band 6                  & tbc                        & $\surd$                & tbc                     & $\surd$  \\
  {\it HST} Cycle 24           & {\footnotesize ACS/WFC3} &{\footnotesize ACS/WFC3}    & {\footnotesize ACS/WFC3}    & {\footnotesize ACS/WFC3} \\
                                       & {\footnotesize {\bf obtained}}  & {\footnotesize {\it pending}}   & {\footnotesize {\it pending}}  & {\footnotesize {\it pending}} \\
 &&&& \\
JWST target visibility (Start) & 2019-04-01    & 2019-05-08    & 2019-05-22   & 2019-06-07  \\ 
JWST target visibility (End)  & 2019-05-07    & 2019-07-01     & 2019-07-15   & 2019-07-29   \\ 
 &&&& \\
\hline\hline
      \end{tabular}
\caption{
Our four Extremely Red Quasar targets. All four quasars were first
identifired in Ross et al. (2015).  Values of e.g. REWs, FWHMs are
from Zakamska et al. (2016) and Hamann et al. (2017).
}
\label{tab:targets} 
  \end{center}
\end{table}



\smallskip \smallskip
\noindent
%% Describe the targets and observational modes to be used. Quantitative
%% estimates must be provided of the accuracy required to achieve key
%% science goals. 
%%
%% Proposers must demonstrate that all observations can
%% execute in the first 5 months of Cycle 1 (planned to be from April to
%% August 2019), and that a substantive subset of the observations are
%% accessible in the first 3 months. This description should also include
%% the following::

\begin{enumerate}[label=\alph*]
    \item{Plan for Alternative Targets: }
    \item{There are no Special Observational Requirements for our ERS ERQ.} 
    \item{We are not inducing Coordinated Parallels observations.}
    \item{The are no duplicated observations.} 
\end{enumerate}

\section{MIRI Overview}
The JWST Mid-Infrared Instrument (MIRI) provides imaging and
spectroscopic observing modes from 4.9 to 28.8 μm (Wright et al. 2015,
Rieke et al. 2015). MIRI offers a very broad range of observing modes,
including: imaging; low-resolution slitted and slitless spectroscopy;
medium-resolution integral field unit (IFU) spectroscopy and
coronagraphy.  We will be utilising the medium-resolution integral
field unit (IFU) and spectroscopy mode.  Medium-resolution
spectroscopy observing mode has a wavelength coverage of
4.9–28.8$\mu$m, a Field of view from 3.9''x3.9'' (for Channel 1) to
7.2''x7.9'' (for Channel 4), a pixel scale of 0.196–0.273 ''/pixel and
a resolving power of $R=\lambda / \Delta lambda = 1550-3250$.  The
FWHm is 2 pix at 6.2$\mu$m (where FWHM = 0.314" $\times (\lambda / 10
\mu$m) for $\lambda > 8\mu$m.)

\smallskip \smallskip
\noindent
The major optical elements in the MRS include 2 gratings/dichroic
wheels and 4 integral field units (IFUs). The MRS also has 2
mid-infrared detectors of the same type used in the imager.  MRS IFUs
split the field of view into spatial slices, each of which produces a
separate dispersed "long-slit" spectrum. Post-processing produces a
composite 3-dimensional (2 spatial and one spectral dimension) data
cube combining the information from each of these spatial slices.  MRS
operations have been designed to allow for efficient observations of
point sources, compact sources, and fully extended sources. The
observer will have control over 3 primary variables: (1) wavelength
coverage, (2) dithering pattern, and (3) detector read out mode and
exposure time (via the number of frames and integrations).

\smallskip \smallskip
\noindent
{\bf \underline{MRS dithering}:} 
The 4 channels of the MRS each cover an overlapping but distinct
region of the JWST focal plane. (see details on the MRS field of view,
coordinate systems, and pointing origin).  The spatial point spread
function (PSF) seen by the imager slicers is undersampled by design,
as is the spectral line spread function (LSF) sampled by the detector
pixels.  Full sampling in both spatial and spectral dimensions
therefore requires that objects be observed in at least 2 (and ideally
4) dither positions that include an offset in both the along-slice and
across-slice directions.

A variety of different dither patterns are
offered that optimize observations for a variety of different
considerations: Point source or extended source observations
(prioritizing PSF separation between successive exposures, or large
common field across all exposures) Spatial sampling at specific
wavelengths or at all wavelengths.  Number of dither locations (2 or
4) Standard or inverted dither orientation Details on the available
patterns can be found at the MIRI MRS Dithering article.  Information
about mosaicing options can be found on the MIRI MRS Mosaics article.

JWST MIRI's ``Slow mode'' readout pattern offers fewer detector
artifacts and slightly lower detector noise than the "fast mode",
making it a good choice for faint source medium-resolution
spectroscopy where the sky backgrounds are very low. This is 
exactly what we want for our ERQ observations. 


\section{MIRI Observing Modes}
%    \subsection{MIRI Imaging}
  %  \subsection{MIRI Coronagraphic Imaging}
    %\subsection{MIRI Low-Resolution Spectroscopy}
    \subsection{MIRI Medium-Resolution Spectroscopy}
    Our observational set-up is:
         \begin{itemize}
           \item MIRI MRS; 
           \item Full spectral coverage; thus we will use all three different spectral settings, SHORT (A), MEDIUM (B), and LONG (C);  
           \item Since with the ERQs we are likely to be observing
             either point sources or compact sources, we choose to use the point
             source optimized, ``4-point ALL'' dither pattern (4PTALL
             dither). According to the pre-flight expected relative performance of
             MRS dither patterns (Figure 3,
             https://jwst-docs.stsci.edu/display/JTI/MIRI+MRS+Dithering) this is
             the only MIRI MRS dither patterns that guarantees ``GOOD''
             (i.e. half-integer) sampling throughout the common field of view
             across all four channels.
           \item We are interested in only a single object per point, so no mosaicking is necessary.  
           \item{MIRI Detector Readout mode:: SLOW}
             \item{Subarray is FULL (fixed for MRS).}
         \end{itemize}
         
\smallskip \smallskip 
\noindent
We take the ``core ERQ'' SED that is given in Hamann et al. (2017) and
is fully representative of the ERQ population at large.  The file
(found on the GitHub) of {\tt core\_ERQ\_SED\_notLog.dat} is used
here.  We normalize this SED at a wavelength of 23$\mu$m to a source
flux density of 5mJy, again very representative of the WISE
W3/4-detected ERQ population given in Ross et al and Hamann et al.

\smallskip \smallskip 
\noindent
At this stage we {\it do not} include any emission (or absorption)
lines since...  In the ETC we assume the Shape of the source is Point.
Other notes, using the ETC, include having: Medium Backgrounds; `IFU
Nod In Scene'; Aperture location Centered on source; Aperture radius
of 0.3''; Nod position in scene of $X=Y=0.5''$.  Our JWST ETC Workbook
has \href{https://jwst.etc.stsci.edu/workbook.html?wb_id=7474}{{\tt wb
ID 7474}}.



\begin{table}
\begin{center}
\begin{tabular}{|| l | r | r | r ||}
  \hline\hline
  &&& \\
  Instrument 	& 	Wavelength  & Science          & \multirow{ 2}{*}{SNR} \\
  Setup	        &    	       of Slice  & exp time / s  &        \\
  &&& \\
  \hline
  &&& \\
 Ch1 SHORT & 5.32  &  4778.00 & 44.67 \\
 Ch2 SHORT & 8.20  &  4778.00 & 91.53 \\
 Ch3 SHORT & 12.00  &  4778.00 & 105.07 \\
 Ch4 SHORT & 18.50 &  4778.00 & 38.37 \\
 Ch1 MEDIUM & 6.00  &  4778.00 & 61.69 \\
 Ch2 MEDIUM & 9.20  &  4778.00 & 105.94 \\
 Ch3 MEDIUM & 14.10  &  4778.00 & 125.79 \\
 Ch4 MEDIUM & 22.80  &  4778.00 & 19.84 \\
 Ch1 LONG & 6.90  &  4778.00 & 74.03 \\
 Ch2 LONG & 10.70  &  4778.00 & 117.20 \\
 Ch3 LONG & 16.90  &  4778.00 & 117.22 \\
 Ch4 LONG & 26.50  &  4778.00 & $^{a}$0 \\
  &&& \\
  \hline\hline
\end{tabular}
\caption{Summary of our MRS Instrument Setup, 
  operating wavelengths of the slices and exposure 
  times and SNR. 
  %% 
  % No wavelength overlap between source_spectra [0.42, 23.32] and instrument [23.95, 28.45].
}
\label{tab:ETC_calcs} 
\end{center}
\end{table}

Each object has a Science Duration of 28668 seconds and a total time charge of 38250s 
for a total Science Duration of  15.93 hours and time charge of 24.04 hours. 
Our ATP Time calculations
\begin{table}
\begin{center}
\begin{tabular}{|| l | r | r ||}
  \hline\hline
  && \\
  Object   	& 	Science           & Total Time          \\
         	        &      Duration        & Charged \\
  && \\
  \hline
  && \\
  J0834+0159  &       14340                & 21630   \\
  J1232+0912 &     28668         & 38250 \\
J2215-0056&      28668  &         38250 \\
J2323-0100 &      28668   &       38250 \\
  && \\
  \hline\hline
\end{tabular}
\caption{Summary of our MRS Instrument Setup, 
  operating wavelengths of the slices and exposure 
  times and SNR. 
  %% 
  % No wavelength overlap between source_spectra [0.42, 23.32] and instrument [23.95, 28.45].
}
\label{tab:ETC_calcs} 
\end{center}
\end{table}


\section{MIRI Overview}
\href{https://jwst-docs.stsci.edu/display/JTI/Mid-Infrared+Instrument\%2C+MIRI}{https://jwst-docs.stsci.edu/display/JTI/Mid-Infrared+Instrument\%2C+MIRI}

\section{MIRI Observing Modes}
https://jwst-docs.stsci.edu/display/JTI/MIRI+Observing+Modes
    \subsection{MIRI Medium-Resolution Spectroscopy}
    https://jwst-docs.stsci.edu/display/JTI/MIRI+Medium-Resolution+Spectroscopy

\section{MIRI Instrumentation}
    \subsection{MIRI Spectroscopic Elements}
    \subsection{MIRI Detector Overview}
        \subsubsection{MIRI Detector Readout Overview}
          \begin{itemize}
            \item{MIRI Detector Readout Slow}
            \item{MIRI Detector Readout Fast}
          \end{itemize}
         \subsubsection{MIRI Detector Performance}
         \subsubsection{MIRI Detector Subarrays}


\section{MIRI Operations}
    \subsection{MIRI Target Acquisition Overview}
    https://jwst-docs.stsci.edu/display/JTI/MIRI+Target+Acquisition+Overview
        %\subsubsection{MIRI Coronagraphic Imaging Target Acquisition}
        %\subsubsection{MIRI LRS Target Acquisition}
        \subsubsection{MIRI MRS Target Acquisition}
        https://jwst-docs.stsci.edu/display/JTI/MIRI+MRS+Target+Acquisition
        Observations with the MIRI MRS IFU may often require a target acquisition
        (TA) procedure, especially at the shortest wavelengths. The required
        pointing precision for the MIRI MRS is 90 mas (1$\sigma$ radial),
        which is approximately the half width of a slice at the shortest
        wavelength. This required accuracy limits the spacecraft move between
        the TA region in the imager and the center of the IFU to <50"
        (Sivaramakrishnan et al. 2006).  The TA may be achieved with the FND,
        F560W, F1000W and F1500W on either the target or a suitable offset
        that is <50". The TA centroiding procedure loses accuracy if the
        pixels are saturated so a brightness limit (Table 1) must be
        considered for the target.
        %\subsubsection{MIRI Bright Source Imaging Target Acquisition}

    \subsection{MIRI Dithering Overview}
    https://jwst-docs.stsci.edu/display/JTI/MIRI+Dithering+Overview
        %% \subsubsection{MIRI Imaging Dithering}
        %% \subsubsection{MIRI Coronagraph Imaging Dithering}
        %% \subsubsection{MIRI LRS Dithering}
        \subsubsection{MIRI MRS Dithering}
        % https://jwst-docs.stsci.edu/display/JTI/MIRI+MRS+Dithering
        We opt for the 4-point ALL dither pattern, point source
        optimized.  As noted in the preparation pages, this provides robust
        performance at all wavelengths and adequate point source separation in
        all channels such that dedicated background observations are not
        required. It is also the only dither where spatial sampling is ``GOOD’
        (i.e. half-integer sampling) throughout the common field of view
        across all four channels.

        \subsubsection{MIRI MRS Dedicated Sky Observations}
        https://jwst-docs.stsci.edu/display/JTI/MIRI+MRS+Dedicated+Sky+Observations

%\subsection{MIRI Mosaics Overview}
%\subsubsection{MIRI Imaging Mosaics}
%\subsubsection{MIRI MRS Mosaics}

%\subsection{MIRI Time Series Observations}
%\subsubsection{MIRI LRS Time Series Observations}
%
%\section{MIRI Predicted Performance}
%\subsection{MIRI Bright Source Limits}
%\subsection{MIRI Sensitivity}


\smallskip \smallskip
\noindent
Specifying JWST Position Angles, Ranges, and Offsets.\\
https://jwst-docs.stsci.edu/display/JPP/Specifying+JWST+Position+Angles\%2C+Ranges\%2C+and+Offsets\\
https://jwst-docs.stsci.edu/display/JPP/JWST+APT+Special+Requirements\\
https://jwst-docs.stsci.edu/display/JSP/JWST+Observing+Overheads+and+Time+Accounting+Policy\\


\section{APT, Overheads and Smart Accounting.}





%%\Huge \huge \LARGE \Large \large \normalsize (default) \small \footnotesize \scriptsize \tiny
